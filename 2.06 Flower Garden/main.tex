\documentclass{article}

% Remember to also load the algostyle.sty file into your project.
\usepackage{algostyle}

% Insert new packages here.

\begin{document}
\begin{question}
Alice is planting $n_1$ flowers $f_1, \dots, f_{n_1}$ among $n_2$ rectangular gardens $\mathcal G_1, \dots, \mathcal G_{n_2}$. Bob's task is to determine which flowers belong to which gardens. Alice informs Bob that no two gardens overlap; therefore, if a flower belongs to a garden, then the flower belongs to {\em exactly} one garden and a garden can contain multiple flowers. If a flower $f_i$ does not belong to any garden, then Bob returns an {\em undefined} garden for $f_i$.

Each garden $\mathcal G_i$ is given by a pair of points $\mathcal G_{BL}[i]$ and $\mathcal G_{TR}[i]$, representing the bottom left and top right corners of the garden respectively. Each flower is represented by a point $F[i]$ representing its location. Let $n = n_1 + n_2$.

\begin{figure}[H]
    \centering
    \begin{tikzpicture}

        \fill[applegreen!20, draw = black] (0, 0) rectangle (2, 1);
        \fill[applegreen!20, draw = black] (2.5, -0.2) rectangle (3.5, 1.5);
        \fill[applegreen!20, draw = black] (0.5, -0.8) rectangle (2.25, -0.2);
        \fill[applegreen!20, draw = black] (3, -0.5) rectangle (3.75, -2);
        
        \node[minimum size = 3pt, inner sep = 0pt] at (1, 0.5) {$\twemoji{sunflower}$};
        \node[minimum size = 3pt, inner sep = 0pt] at (3.25, 1) {$\twemoji{sunflower}$};
        \node[minimum size = 3pt, inner sep = 0pt] at (1.5, -0.5) {$\twemoji{sunflower}$};
        \node[minimum size = 3pt, inner sep = 0pt] at (1.2, -0.55) {$\twemoji{sunflower}$};
        \node[minimum size = 3pt, inner sep = 0pt] at (3.2, -1) {$\twemoji{sunflower}$};
    \end{tikzpicture}
    \caption{{\em A collection of $n_1 = 5$ flowers and $n_2 = 4$ gardens}.}
    % \label{fig:enter-label}
\end{figure}

Formally, you are given three arrays:
\begin{itemize}
    \item $\mathcal G_{BL} = [(x_1, y_1), \dots, (x_{n_2}, y_{n_2})]$, where $\mathcal G_{BL}[i] = (x_i, y_i)$ represents the bottom left point of garden $\mathcal G_i$,
    \item $\mathcal G_{TR} = [(x_1, y_1), \dots, (x_{n_2}, y_{n_2})]$, where $\mathcal G_{TR}[i] = (x_i, y_i)$ represents the top right point of garden $\mathcal G_i$, and
    \item $F = [(x_1, y_1), \dots, (x_{n_1}, y_{n_1})]$, where $F[i] = (x_i, y_i)$ represents the location of flower $f_i$.
\end{itemize}

You are not guaranteed that a flower belongs to a garden; it is possible that a flower is planted in none of the gardens. Your goal is to return the garden that a flower $f_i$ belongs to (if any), for {\em each} $f_i$.

\begin{itemize}
    \item For example, in the diagram above, if the flower ordering is sorted by its $x$-coordinate and the rectangular gardens are sorted by their $x$-coordinate of $\mathcal G_{BL}$, then we return \[f_1: \mathcal G_1, \quad f_2: \mathcal G_2, \quad f_3: \mathcal G_2, \quad f_4: \mathcal G_4, \quad f_5: \mathcal G_3.\]
\end{itemize}

\begin{enumerate}[label = (\alph*)]
    \item We first solve the special case where all of the gardens intersect with a horizontal line. Describe an $O(n \log n)$ algorithm to determine which flowers belong to which gardens (if such a garden exists).

    \begin{figure}[H]
        \centering
        \begin{tikzpicture}
            \fill[applegreen!20, draw = black] (0, 0) rectangle (2, 1);
            \fill[applegreen!20, draw = black] (2.5, -0.2) rectangle (3.5, 1.5);
            \fill[applegreen!20, draw = black] (3.75, 0.3) rectangle (4.5, 0.7);
            \fill[applegreen!20, draw = black] (4.75, -0.5) rectangle (5.75, 1.35);
    
            \draw[dotted] (-0.5, 0.5) -- (6.25, 0.5);
            % \draw (0.5, -0.8) rectangle (2.25, -0.2);
            % \draw (3, -0.5) rectangle (3.75, -2);
    
            \node[, minimum size = 3pt, inner sep = 0pt] at (1, 0.5) {$\twemoji{sunflower}$};
            \node[minimum size = 3pt, inner sep = 0pt] at (3.25, 1) {$\twemoji{sunflower}$};
            \node[minimum size = 3pt, inner sep = 0pt] at (4, 0.7) {$\twemoji{sunflower}$};
            \node[minimum size = 3pt, inner sep = 0pt] at (3.25, 0) {$\twemoji{sunflower}$};
            \node[minimum size = 3pt, inner sep = 0pt] at (1.2, 0.25) {$\twemoji{sunflower}$};
            \node[minimum size = 3pt, inner sep = 0pt] at (5, -0.1) {$\twemoji{sunflower}$};
        \end{tikzpicture}
        \caption{{\em A collection of $n_1 = 6$ flowers and $n_2 = 4$ gardens that intersect with a horizontal line.}}
        % \label{fig:enter-label}
    \end{figure}

    {\bfseries Hint.} {\em What do you know about two adjacent gardens if they have to intersect with a horizontal line?}

    \item We now remove the assumption that every garden intersects with a horizontal line. Describe an $O(n (\log n)^2)$ algorithm to determine which flowers belong to which gardens (if such a garden exists).
\end{enumerate}
\end{question}

\begin{rubric}
\begin{itemize}
    \item Your solution should clearly outline which subpart you're answering.

    \item If your solution is a simple modification of a previous solution, you do not need to restate the solution; you can simply refer to the previous part in your solution.

    \item As usual, you should argue the correctness of the algorithm and its time complexity.

    \item This task will form part of the portfolio.
    \item Ensure that your argument is clear and keep reworking your solutions until your lab demonstrator is happy with your work.
\end{itemize}
\end{rubric}

\begin{solution}
\begin{enumerate}[label = (\alph*)]
    \item If there exists a line such that it intersects all gardens, then there exists no horizontal overlap between any of the gardens.
    Sort the gardens based on the x-coordinate of their bottom-left corner in increasing order. This takes $O(n\log n)$ time.
    
    For each flower, perform a binary search for the first garden from the left such that the flower is to the right of the bottom-left corner of it.
    Since there are no horizontal overlaps, the flower is either in this garden, or it is not in any garden at all. Performing this check is $O(1)$.
    Since the number of flowers and gardens are both at maximum $n$, this full process of finding the flowers' corresponding gardens takes $O(n\log n)$.

    \item We will approach the whole problem using a divide and conquer approach, each time halving the total number of flowers and gardens.
    First, we will preprocess the input. Sort gardens in order of increasing x-coordinate of bottom-left corner, and then stably sort in order of increasing y-coordinate of bottom-left corner. This preprocessing stage should have a time cost of $O(n\log n)$.
    Now we can begin the divide and conquer algorithm to solve the following question:

    \textbf{Problem Statement}

    Each garden is an object with the following attributes:
    \begin{itemize}
        \item Initial index before preprocessing
        \item Coordinates of bottom-left corner
        \item Coordinates of top-right corner
    \end{itemize}
    Each flower is an object with the following attributes:
    \begin{itemize}
        \item Initial index before preprocessing 
        \item Coordinates
    \end{itemize}

    You are given an array $G$ of gardens, sorted by the y-coordinate of their bottom-right corners.
    You are also given an array $F$ of flowers, sorted by their y-coordinate.
    Let $n$ be equal to the sum of the number of flowers and gardens.

    For each flower, determine the garden which it belongs to.
    Return this solution as a list of pairs $(F_i, G_j)$ of flowers and the garden each belongs to.

    \pagebreak

    \textbf{Divide}

    The first step of the divide step is to find the median of the two sorted arrays $G$ and $F$ in $O(n)$ time.
    We can do this simply by traversing the $G$ and $F$ arrays concurrently with one pointer each, 
    with each iteration determining which pointer to increment by comparing the y-coordinate of the entities,
    and terminating when a total of $n/2$ pointer increments have been performed. 
    From this we can divide $G$ into two subarrays $G_L$ and $G_R$, which are the gardens to the 
    left and the right of the median entity respectively. These subarrays are sorted in the same fashion that $G$ is.
    We can also divide $F$ into two sorted subarrays $F_L$ and $F_R$, which are the the flowers to the left and the right of the median entity respectively.
    Note that $G_L + F_L = G_R + G_L = n/2$.
    Finding and copying these sorted subarrays takes $O(n)$ time.

    Imagine that a horizontal line is drawn at the median entity (bottom-right corner if it is a garden).
    All gardens in $G_R$ are completely above this line. However the gardens in $G_L$ are either completely below the line, or intersecting with this line.
    We can split $G_L$ into two subsequences, one containing all the gardens completely below the line, and one containing all the gardens intersecting with the line.
    Both these subsequences can be extracted into arrays $G_B$ (B for "completely below") and $G_I$ (I for intersecting), which are both sorted in the same fashion as $G$.
    Since $G$ is already sorted in this way, constructing $G_B$ and $G_I$ takes $O(n)$ time.

    We can now divide this problem into two problems:
    \begin{itemize}
        \item Sorted flower array $F_L$, sorted garden array $G_B$.
        \item Sorted flower array $F_R$, sorted garden array $G_R$.
    \end{itemize}
    Both of these subproblems have an $n'\leq n/2$.

    Note that the gardens in $G_I$ are left out of the division. We will use this array again during the combine step.

    \pagebreak

    \textbf{Base Case}

    The base case is when the total number of entities in the problem becomes one.
    This is guaranteed to happen as the problem size halves with each division.
    If there is one flower, then return the flower paired with \textsc{NULL}. 
    If there is one garden, then return an empty list.

    \textbf{Combine}

    Both subproblems will return an array of pairs of flowers and their corresponding garden.
    Concatenate these two lists to form $L$. Perform a sweep of this list to find all the flowers which do not have a corresponding garden yet,
    and extracting (deleting them from the original list) these flowers into an array $F_N$ (N for "not in garden yet").
    We can perform the solution from part (a) on the array of flowers $F_N$ and the sorted array $G_I$.
    This takes $O(n\log n)$, and returns another array of flowers paired with their corresponding gardens.
    Concatenate this list with $L$. Return this final list.

    \textbf{Time Complexity}

    The time cost satisfies the the recurrence $T(n)=2T(\frac{n}{2}) + n\log n$.
    According to the Master Theorem, $T(n)\in O(n(\log n)^2)$.
    % add a bit justifying the time complexity using the master theorem


\end{enumerate}
\end{solution}
\end{document}