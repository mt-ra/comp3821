\documentclass{article}

% Remember to also load the algostyle.sty file into your project.
\usepackage{algostyle}

% Insert new packages here.

\begin{document}
\begin{question}
You are given a string $S$ of $n$ characters and another string $T$ of $m$ characters such that $m \leq n$. A {\em subsequence} $S'$ of $S$ is any (not necessarily contiguous) sequence of characters within $S$. For example, a subsequence of $S = abcde$ is $S' = abd$. A {\em supersequence} of $S'$ is any sequence of characters that contains $S'$ as a subsequence. For example, $S = bacedf$ is a supersequence of $S' = bcd$. Similarly, a {\em superstring} is a contiguous supersequence.

You want to find the length of the longest subsequence of $S$ that appears as a prefix of $T$. For example, if $S = abcdefgh$ and $T = bcdghf$, then your algorithm should return 5.

\begin{enumerate}[label = (\alph*)]
    \item Let $T'$ be a prefix of $T$. Show that:
    \begin{itemize}
        \item If $T'$ is a subsequence of $S$, then any substring of $T'$ is also a subsequence of $S$.
        \item If $T'$ is not a subsequence of $S$, then any superstring of $T'$ is not a subsequence of $S$.
    \end{itemize}

    \item For a given string $A$ of $n$ characters and another string $B$ of $m$ characters (with $m \leq n$), assume that there is an $O(f(n))$ algorithm that decides if $B$ is a subsequence of $A$. Using this algorithm, describe an $O(f(n) \log m)$-time algorithm to compute the length of the longest subsequence of $S$ that appears as a prefix of $T$.
\end{enumerate}
\end{question}

\begin{rubric}
\begin{itemize}
    \item Your argument should prove both results. You can use any result you previously proved in your argument.
    
    \item You should justify why your algorithm is correct and why they run in the allocated time complexities (or faster!).

    \item This task will form part of the portfolio.
    \item Ensure that your argument is clear and keep reworking your solutions until your lab demonstrator is happy with your work.
\end{itemize}
\end{rubric}

\begin{solution}
\begin{enumerate}[label = (\alph*)]
    \item If $T'$ is a subsequence of $S$, and $T''$ is $T'$ but with any number of characters removed from anywhere, then $T''$ will also be a subsequence of $S$.
    This is because the letters in $T''$ will also exist in that order in $S$. Therefore all substrings of $T'$ will also be subsequences of $S$.

    If $T'$ is not a subsequence of $S$, and $T''$ is $T'$ but with any number of characters inserted at any position,
    then $T''$ will also not be a subsequence of $S$. This is because if $T''$ were a subsequence of $S$, then $T'$ would also be a subsequence of $S$ due to the previous part.
    This is a contradiction, therefore any superstring of $T'$ will also not be a subsequence of $S$.

    \item Define the decision problem ``is the prefix of $T$ of length $X$ a subsequence of $S$'' for integer parameters $X$.
    Let $D(x)$ be the outcome of the decision problem when $X=x$,
    with $0$ for false and $1$ for true. We can determine $D(x)$ for any $x$ in $O(f(n))$ time.
    
    Suppose that $D(k)=1$ for some $k$. By definition, the prefix formed by the first $k$ characters of $T$ is a subsequence of $S$. 
    From part (a), every substring, including the prefix formed by the first $(k-1)$ characters of $T$, are also subsequences of $S$. Therefore if $D(k)=1$ then $D(k-1)=1$.

    Therefore performing a binary search on the decision problem $f$ for the lastest value $k$ for which $f(k)=1$, 
    will solve this problem. The time complexity is $O(f(n)\log (n))$.
\end{enumerate}
\end{solution}
\end{document}