\documentclass{article}
\usepackage{algostyle}

\begin{document}
% \task[regular]{Interpolation}

\begin{question}
\emph{Interpolation} is the operation of fitting a curve to a set of points. That is, given a set of points $\left\{(x_1, y_1), \dots, (x_k, y_k)\right\}$, interpolation finds a function $f$ such that $f(x_1) = y_1, \dots, f(x_k) = y_k$. The graph of $f$ passes through all the given points.

Throughout this problem, we will assume that arithmetic operations on real and complex numbers take constant time.

\begin{enumerate}[label=(\alph*)]
    \item (optional) How would you fit a polynomial to \emph{any} $2n+1$ points in $\Theta(n^2)$ time?

    \item How would you fit a polynomial to $2n+1$ values at \emph{specifically} the $(2n+1)$-th roots of unity in $\Theta(n \log n)$ time?

    \item Suggest how to fit a \emph{Laurent polynomial} \[ \sum_{k=-n}^n c_k x^k, \] instead of a polynomial \[ \sum_{k=0}^{2n} c_k' x^k \] to the values at the roots of unity.
    
    \item Show that if the values specified at the $2n+1$ roots of unity are real, then $c_k = \overline{c_{-k}}$ for all $k$, where $\overline{\cdot}$ denots complex conjugate.
\end{enumerate}

The true power of this technique comes from replacing the variable in a polynomial expression with a complex exponential.

\begin{enumerate}[label=(\alph*)]
    \setcounter{enumi}{4}
    \item Show how by setting $x = e^{it}$, we can rewrite \[ \sum_{k=-n}^n c_k x^k \] in the form of a \emph{trigonometric polynomial} \[ A_0 + \sum_{k=1}^n A_k \cos(kt + \phi_k), \] where all the coefficients $A_k$ and $\phi_k$ are complex numbers given by
    \[ A_k = 2\sqrt{c_kc_{-k}}, \;\; \phi_k = -\tan^{-1}\left( \frac{i(c_k-c_{-k})}{(c_k+c_{-k})} \right) \]
    \item Therefore describe how to fit a trigonometric polynomial to $2n+1$ \emph{equally spaced} points $0, d, 2d, \ldots$ in $\Theta(n \log n)$ time.
    \item Using the results from parts (d) and (e), show that if the values at each of the $2n+1$ equally spaced points are real, then the trigonometric polynomial interpolation is also a real valued function (i.e. $\phi_k$ and $A_k$ are real valued for all $k$).
    \item This methodology allows us to interpolate any sequence of equally spaced points with a periodic function in the form of a finite series of several sinusoids.
    \begin{itemize}
        \item What distinguishes the sinusoids arising from each term of the trigonometric polynomial?
        \item What do the coefficients $A_k$ and $\phi_k$ represent?
    \end{itemize}
    \item Suggest how this might be applied to:
    \begin{enumerate}[label=(\roman*)]
        \item processing audio to remove unwanted low pitch sound (such as wind);
        \item analysing time series in finance or climatology;
        \item (optional, hard) remove the \emph{aliasing} effect that arises when an image is reconstructed from samples;
        \item (optional, very hard) compressing an image using JPEG.
    \end{enumerate}
\end{enumerate}
\end{question}

\begin{rubric}
\begin{enumerate}
    \item State a method to find any algebraic expression (not necessarily the coefficient representation) for the fitted polynomial. The question is optional, so you can delve into as much or as little as you like.
    \item State a method to convert from this particular value representation to the coefficient representation.
    \item Discuss the differences in fitting a `polynomial' with negative exponents as compared to a regular polynomial.
    \item Start with the inverse DFT expression for $c_k$ and $c_{-k}$ and show that they are conjugates.
    \item Perform the substitution and continue manipulating the expression until you reach the desired form.
    \item Describe how to transform the input in this scenario into inputs suitable for the algorithm developed in the previous parts, and how to `undo' this transformation to recover the solution.
    \item Combine the results from (d) and (e) to mathematically show that all of the coefficients are real valued.
    \item It's sufficient to just directly answer the two dot points.
    \item Only a very cursory explanation is needed here. We don't expect you to know the ins and outs of these applications, but they are starting points for your own investigation into perhaps the most important modern algorithm.
\end{enumerate}

Expected response length:
\begin{enumerate}
    \setcounter{enumi}{1}
    \item one sentence
    \item a short paragraph
    \item a few lines of mathematical working
    \item a short passage of mathematical working
    \item a paragraph
    \item a few lines of mathematical working
    \item a sentence for each dot point
    \item a sentence or short paragraph each for (i) and (ii)
\end{enumerate}
\end{rubric}

\pagebreak

\begin{solution}

    (a)

    (b) The Inverse DFT solves this problem, where we just perform an FFT, 
    but replace the $(2n+1)$th root of unity with its complex conjugate, and scale the 
    resulting output values by $1/(2n+1)$.

    (c) Let $\omega$ be a $(2n+1)$-th root of unity.

    Note that for positive integers $i$, it is true that $\omega^{n+i}=1\times\omega^{n+1}=(\omega^{2n+1})^{-1}\times\omega^{n+i}=\omega^{-n-1+i}$.
    Suppose that we have a polynomial $P$ fitted to 2n+1 values at the $(2n+1)$-th roots of unity.
    $$P(x)=\sum_{k=0}^{2n}c'_kx^k.$$
    Using our observation from above, we can deduce that if $x$ is a $(2n+1)$-th root of unity,
    \begin{align*}
        P(x) &= \sum_{k=0}^{2n}c'_kx^k\\
        &= \sum_{k=n+1}^{2n}c'_kx^k + \sum_{k=0}^{n}c'_kx^k\\
        &= \sum_{p=-n}^{-1}c'_kx^{p+2n+1} + \sum_{k=0}^{n}c'_kx^k \qquad (\text{Let } k=p+2n+1)\\
        &= \sum_{p=-n}^{-1}c'_kx^{p} + \sum_{k=0}^{n}c'_kx^k \qquad (\text{Since } \omega^{n+i}=\omega^{-n-1+i})\\
    \end{align*}

    Therefore let $c_k=c'_k$ for $0\leq k \leq n$. Let $c_k=c'_{k+2n+1}$ for $-n\leq k \leq -1$.

    (d) Note that the relationship between a fitted polynomial and its corresponding Laurent polynomial is bijective,
    as the Laurent polynomial is simply the polynomial but with the coefficients rearranged.
    Suppose that two Laurent polynomials of degree $N$ pass through the same $N + 1$ points. 
    Their corresponding polynomials will therefore also pass through the same $N + 1$ points.
    Therefore their corresponding polynomials are identical.
    Since this correspondence is bijective, this means that the Laurent polynomials must also be identical.

    For any $(2n+1)$-th roots of unity $\omega$, it is true that $\omega \cdot \overline{\omega}=1$ and thus $\overline{\omega}=\omega^{-1}$.
    Since the value of the Laurent polynomial evaluated at any $\omega$ is real, this means that $P(\omega)=\overline{P(\omega)}$.
    \begin{align*}
        \sum_{k=-n}^{n}c_k\omega^k &= \overline{\sum_{k=-n}^{n}c_k\omega^k}\\
        &= \sum_{k=-n}^{n}\overline{c_k}(\overline{\omega})^k \\
        &= \sum_{k=-n}^{n}\overline{c_k}\omega^{-k} \\
        \sum_{k=-n}^{n}c_k\omega^k &= \sum_{k=-n}^{n}\overline{c_{-k}}\omega^{k}
    \end{align*}

    These last two Laurent polynomials have a degree of $2n$ and are equal for all $(2n+1)$-th roots of unity $\omega$, and therefore must be identical.
    Therefore $c_k=\overline{c_{-k}}$ for all $k$.

    \pagebreak
    
    (e) For any positive real number $n$ and sequence $\{c_{-n}, c_{-(n-1)},\dots, c_{n-1}, c_n\}$,

    For any $1\leq k \leq n$,
    \begin{align*}
        \cos\left(\tan^{-1}\left(i\frac{c_k-c_{-k}}{c_k+c_{-k}}\right)\right) &= \frac{1}{\sqrt{1+\left(i\frac{c_k-c_{-k}}{c_k+c_{-k}}\right)^2}}\\
        &= \frac{c_k+c_{-k}}{\sqrt{{(c_k+c_{-k})^2-(c_k-c_{-k})^2}}}\\
        &= \frac{c_k+c_{-k}}{2\sqrt{c_kc_{-k}}}
    \end{align*}

    \begin{align*}
        \sin\left(\tan^{-1}\left(i\frac{c_k-c_{-k}}{c_k+c_{-k}}\right)\right) &= \frac{\left(i\frac{c_k-c_{-k}}{c_k+c_{-k}}\right)}{\sqrt{1+\left(i\frac{c_k-c_{-k}}{c_k+c_{-k}}\right)^2}}\\
        &= \frac{c_k+c_{-k}}{2\sqrt{c_kc_{-k}}}\left(i\frac{c_k-c_{-k}}{c_k+c_{-k}}\right)\\
        &= i\frac{c_k-c_{-k}}{2\sqrt{c_kc_{-k}}}
    \end{align*}



    Therefore,
    \begin{align*}
        C(t) &:= c_0+\sum_{k=1}^{n}\left(2\sqrt{c_kc_{-k}}\cos\left(kt-\tan^{-1}\left(i\frac{c_k-c_{-k}}{c_k+c_{-k}}\right)\right)\right) \\
        &= c_0+\sum_{k=1}^{n}\left(2\sqrt{c_kc_{-k}}\left(\cos(kt)\cos\left(\tan^{-1}\left(i\frac{c_k-c_{-k}}{c_k+c_{-k}}\right)\right) + \sin(kt)\sin\left(\tan^{-1}\left(i\frac{c_k-c_{-k}}{c_k+c_{-k}}\right)\right)\right)\right)\\
        &= c_0+\sum_{k=1}^{n}\left(2\sqrt{c_kc_{-k}}\left(\cos(kt)\frac{c_k+c_{-k}}{2\sqrt{c_kc_{-k}}} + \sin(kt)i\frac{c_k-c_{-k}}{2\sqrt{c_kc_{-k}}}\right)\right)\\
        &= c_0+\sum_{k=1}^{n}\left(\cos(kt)(c_k+c_{-k}) + i\sin(kt)(c_k-c_{-k})\right)\\
        &= c_0+\sum_{k=1}^{n}\left(c_k(\cos(kt) + i\sin(kt)) + c_{-k}(\cos(kt) - i\sin(kt))\right)\\
        &= c_0+\sum_{k=1}^{n}\left(c_k(e^{it})^k + c_{-k}(e^{it})^{-k}\right)\\
        &= \sum_{k=-n}^{n}c_k(e^{it})^k
    \end{align*}

\end{solution}
\pagebreak
\begin{solution}

    (f) Suppose that we have found a Laurent polynomial $L(x)$ which fits values taken at the $(2n + 1)$-th roots of unity.
    That is, given $\omega$ is the $(2n+1)$-th root of unity with the smallest positive argument, for all $0\leq k \leq 2n$:
    $$L(\omega^k)=v_{k}.$$

    Therefore the trigonometric polynomial,
    $$C(t):=c_0+\sum_{k=1}^{n}\left(2\sqrt{c_kc_{-k}}\cos\left(kt-\tan^{-1}\left(i\frac{c_k-c_{-k}}{c_k+c_{-k}}\right)\right)\right)$$
    will fit those same values, with $x=e^{it}$. That is, for all $0\leq k \leq 2n$:
    $$C\left(k\frac{2\pi}{2n+1}\right)=v_k.$$
    To extend this approach to all sets of evenly spaced points in the form $(0, v_0), (d, v_1), (2d, v_2),\dots$, we can just scale $C$ along the input axis.
    In this case we want a scaling factor of $\frac{d(2n+1)}{2\pi}$.

    Therefore the polynomial 
    $$Q(t) := C\left(\frac{2\pi}{d(2n+1)}t\right)$$
    will successfully fit these evenly spaced points.

    To summarise, we are given the points $(0, v_0), (d, v_1), (2d, v_2),\dots, (2nd, v_{2n})$ to interpolate.
    Perform the strategy performed in part (b) to find a polynomial $P(x)=\sum_{k=0}^{2n}c'_kx^k$ which passes 
    through the points $(\omega^0, v_0), (\omega^1, v_1), \dots, (\omega^{2n}, v_{2n})$.
    This process takes $O(n\log n)$ time.
    
    By rearranging the coefficients as discussed in part (c), 
    we can then find a \\Laurent polynomial $L(x) = \sum_{k=-n}^{n}c_kx^k$ which passes through the \\same points $(\omega^0, v_0), (\omega^1, v_1), \dots, (\omega^{2n}, v_{2n})$.
    This step takes $O(n)$ time. 

    Then we can find the trigonometric polynomial $C(x)$ which passes through \\the points $(0\frac{2\pi}{2n+1}, v_0), (1\frac{2\pi}{2n+1}, v_1), \dots, (2n\frac{2\pi}{2n+1}, v_{2n})$,
    also taking $O(n)$ time. Finally we scale this polynomial into $Q(x)$, which passes through $(0, v_0), (d, v_1), \dots, (2nd, v_{2n})$. This also takes $O(n)$ time.

    Therefore we can interpolate these evenly spaced points in $O(n\log n)$ time, dominated by the inverse DFT stage at the beginning.

    (g) If the output values are real, then using our result from (d):
    \begin{align*}
        A_k &=2\sqrt{c_kc_{-k}}\\
        &=2\sqrt{c_k\overline{c_k}}\\
        &=2\sqrt{|c|}\in \mathbb{R}
    \end{align*}
    \begin{align*}
        \phi_k &= -\tan^{-1}\left(i\frac{c_k-c_{-k}}{c_k+c_{-k}}\right)\\
        &= -\tan^{-1}\left(i\frac{c_k-\overline{c_k}}{c_k+\overline{c_k}}\right)\\
        &= -\tan^{-1}\left(i\frac{2i\Im(c_k)}{2\Re(c_k)}\right)\\
        &= \tan^{-1}\left(\frac{\Im(c_k)}{\Re(c_k)}\right) \in \mathbb{R}
    \end{align*}

    \pagebreak 

    (h) (i) Take samples of the voltage of the sound wave at evenly spaced, finely spaced points in time.
    Use the strategy in (g) to interpolate these points with the sum of sinusoids.
    Get rid of the low frequency sinusoids, and evaluate the value of this new modified trigonometric polynomial at the original points in time.

    (ii) This may be useful to discover periodic trends in finance or climatology that may appear invisible to the human eye.
    These trends can be used to turn a profit.




\end{solution}

\end{document}