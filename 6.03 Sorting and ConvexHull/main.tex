\documentclass{article}

% Remember to also load the algostyle.sty file into your project.
\usepackage{algostyle}

% Insert new packages here.

\begin{document}
\begin{question}
Consider the two pair of problems.
\begin{itemize}
    \item $\compproblem{Sorting}$: {\em Given a set of $n$ integers, the task is to sort the integers in increasing order.}
    \item $\compproblem{ConvexHull}$: {\em Given a set $X$ of $n$ 2-dimensional points, the task is to compute the smallest convex set (i.e. a set that can be formed by joining line segments) that contains all of the points in $X$.}
\end{itemize}

\begin{enumerate}[label = (\alph*)]
    \item Show that there exist a linear-time reduction from $\compproblem{Sorting}$ to $\compproblem{ConvexHull}$.
    \item Hence, deduce that any algorithm which solves $\compproblem{ConvexHull}$ has an $\Omega(n \log n)$ lower bound.

    {\em This means that, in the worst case, an algorithm such as \href{https://en.wikipedia.org/wiki/Graham_scan}{Graham scan} and \href{https://en.wikibooks.org/wiki/Algorithm_Implementation/Geometry/Convex_hull/Monotone_chain}{Monotone chain} is tight.}
\end{enumerate}
\end{question}

\begin{solution}
\begin{enumerate}[label = (\alph*)]
    \item For each integer $k$ in the input of $\compproblem{Sorting}$, 
    insert $(k, k^2)$ into the set $X$, which will be the input for $\compproblem{ConvexHull}$.
    The result is the points in the convex hull in anticlockwise order.
    Scanning through the convex hull for the minimum element, and then marching anticlockwise along the convex hull, 
    will give the input array in sorted order.

    This reduction is linear time as only two linear scans are performed along containers of size $n$.


    \item We already know that sorting can be done in no faster than $O(n\log n)$ time.
    By using the reduction to $\compproblem{ConvexHull}$, we can 
    solve $\compproblem{Sorting}$ in $O(n)$ (the time taken to do the reduction) plus 
    the time taken to do $\compproblem{ConvexHull}$.
    If a general $\compproblem{ConvexHull}$ instance could be solved in any faster than $O(n\log n)$ time, 
    then this would mean that $\compproblem{Sorting}$ could be solved in faster than $O(n\log n)$ time,
    which is impossible. Therefore any algorithm which solves $\compproblem{ConvexHull}$ has a $\Omega(n\log n)$ lower bound.
    % FINISH THIS
\end{enumerate}
\end{solution}
\end{document}