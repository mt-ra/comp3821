\documentclass{article}

% Remember to also load the algostyle.sty file into your project.
\usepackage{algostyle}

% Insert new packages here.

\begin{document}
\begin{question}
Let $F = (V, E, s, t, w)$ be an $(s, t)$-flow network. In this problem, our task is to be able to classify vertices in $V$ by a special kind of property that we will define.

\begin{itemize}
    \item A vertex $v$ is {\em saucy} if $v$ appears on the source side of {\em every} minimum cut; that is, if $(S, T)$ is a minimum cut of $F$, then $v \in S$.

    \item A vertex $v$ is {\em sinky} if $v$ appears on the sink side of {\em every} minimum cut; that is, if $(S, T)$ is a minimum cut of $F$, then $v \in T$.

    \item A vertex $v$ is {\em saunky} if $v$ is neither saucy nor sinky. In other words, there exist at least one minimum cut $(S, T)$ for which $v \in T$ and there exist at least one minimum cut $(S', T')$ for which $v \in S'$.
\end{itemize}

Describe an $O(mn^2)$ algorithm to classify each vertex as either saucy, sinky, or saunky.

{\bfseries Note.} {\em There can exist exponentially many minimum cuts in a flow network, so do not try and enumerate over all possible minimum cuts.}
\end{question}

\begin{solution}

Let $f$ be a max flow.
For all minimum cuts $(S,T)$, $f$ will saturate all edges from $S$ to $T$, 
and there will be no flow across any edges from $T$ to $S$. 
The converse is true, where if there exists a cut $(S,T)$ such that $f$
saturates all edges from $S$ to $T$, and does not have flow across edges from $T$ to $S$,
then the $c(S,T)=|f|$ and hence $(S,T)$ is a minimum cut.
In other words, $(S, T)$ is a minimum cut if and only if there doesn't exist an edge from $S$ to $T$
in the residual graph when the flow is $f$.

Firstly, we want to show that the intersection of two minimum cuts on the source side is also a minimum cut.
In other words we want to show that if $(S_1, T_1)$ and $(S_2, T_2)$ are minimum cuts,
then $(S_1\cap S_2, T_1 \cup T_2)$ is also a minimum cut.
The reason why I am calling it the intersection is because you can represent a cut with just the source-side set $S$,
because $T$ is just $V - S$.

Now let $(S_1, T_1)$ and $(S_2, T_2)$ be two minimum cuts.
Therefore there are no residual edges from $S_1$ to $T_1$, nor are there any residual edges from $S_2$ to $T_2$.
All edges from $S_1\cap S_2$ to $T_1\cup T_2$ are either going from $S_1$ to $T_1$, or from $S_2$ to $T_2$, or both at the same time.
In any case, these edges cannot exist on the residual graph. 
Therefore $(S_1\cap S_2, T_1\cup T_2)$ is also a minimum cut.

Secondly, we want to show that the union $(S_1\cup S_2, T_1 \cap T_2)$ of two minimum cuts $(S_1, T_1), (S_2, T_2)$ on the source side is also a minimum cut.
We can make a similar argument where all edges coming from somewhere else into $T_1\cap T_2$ is either coming from 
$S_1$, or $S_2$, or both. Therefore the cut $(S_1\cup S_2, T_1\cap T_2)$ is also a minimum cut.

Let $I$ be the intersection of all minimum cuts, and $U$ be the union of all minimum cuts. 
Note that both are minimum cuts, $I$ being the minimum cut with the least number of vertices on the source side,
and $U$ being the minimum cut with the most number of vertices on the source side.
All vertices in $I$ are saucy, all vertices in $U - I$ are saunky, and all vertices in $V - U$ are sinky.

First we want to find a max flow $f$. We can do this with Dinic's algorithm, taking $O(mn^2)$ time.
We can find $I$ by performing a BFS on the residual graph from the source. 
Each edge reachable by the BFS is in $I$.
We can find $U$ by performing a BFS on the residual graph from the sink, but with all edges reversed.
Each edge reachable by this BFS is in $U$.
Both of these searches take $O(m + n)\subset O(mn^2)$ time.

Finally you can look through all the vertices, for each one determining if it is in $I$, $U-I$ or $V-U$.
This takes $O(n)$ time. Therefore we can classify all the vertices in $O(mn^2)$ time.


\end{solution}
\end{document}