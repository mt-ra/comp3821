\documentclass{article}

% Remember to also load the algostyle.sty file into your project.
\usepackage{algostyle}

% Insert new packages here.

\begin{document}
\begin{question}
You are given a set $H$ of horizontal line segments and a set $V$ of vertical line segments such that $\lvert H \rvert + \lvert V \rvert = n$. Each horizontal line segment is specified by two $x$-coordinates and one $y$-coordinate $\{x_{i_1}, x_{i_2}, y_i\}$ and each vertical line segment is specified by two $y$-coordinates and one $x$-coordinate $\{y_{i_1}, y_{i_2}, x_i\}$.

{\em In all three subparts, you may assume that all of the $x$ and $y$ coordinates are distinct; in other words, no pair of horizontal lines intersect and no pair of vertical lines intersect.}

\begin{enumerate}[label = (\alph*)]
    \item Firstly, suppose that each vertical line segment is unbounded from above and below. In other words, each vertical line segment stretches from $-\infty$ to $+\infty$ in its $y$-coordinate.

    Describe an $O(n \log n)$ algorithm to return the number of intersections between line segments in $H$ and {\em lines} in $V$.

    \begin{figure}[H]
        \centering
        \begin{tikzpicture}
            \draw[red, thick, latex-latex] (0.5, -1.5) -- (0.5, 1.5);
            \draw[red, thick, latex-latex] (3, -1.5) -- (3, 1.5);
            \draw[red, thick, latex-latex] (1.5, -1.5) -- (1.5, 1.5);
            \draw[red, thick, latex-latex] (2, -1.5) -- (2, 1.5);

            \node[fill = blue, circle, inner sep = 0pt, minimum size = 4pt] (a) {};
            \node[fill = blue, circle, inner sep = 0pt, minimum size = 4pt, right = 2cm of a] (b) {};
            \node[fill = blue, circle, inner sep = 0pt, minimum size = 4pt, below right = 1cm of a] (c) {};
            \node[fill = blue, circle, inner sep = 0pt, minimum size = 4pt, right = 0.75cm of c] (d) {};
            \node[fill = blue, circle, inner sep = 0pt, minimum size = 4pt, above right = 1cm of a] (e) {};
            \node[fill = blue, circle, inner sep = 0pt, minimum size = 4pt, right = 1.75cm of e] (f) {};

            \draw[thick, blue] (a) -- (b);
            \draw[thick, blue] (c) -- (d);
            \draw[thick, blue] (e) -- (f);
        \end{tikzpicture}
        \caption{{\em There are six intersections between the horizontal line segments and the vertical lines.}}
        \label{fig:enter-label}
    \end{figure}

    {\bfseries Hint.} {\em Sort...}

    \item Now, suppose that each vertical line is bounded from above. In other words, each vertical line has a top endpoint but each vertical line stretches downwards to $-\infty$. We call these {\em rays} and each ray can be specified by one $x$-coordinate and one $y$-coordinate $\{x_i, y_i\}$.
    
    Describe an $O(n \left(\log n\right)^2)$ algorithm that returns the number of intersections between line segments in $H$ and vertical {\em rays} in $V$.

    \begin{figure}[H]
        \centering
        \begin{tikzpicture}

            \node[fill = red, circle, inner sep = 0pt, minimum size = 4pt] at (0.5, 1.5) {};
            \node[fill = red, circle, inner sep = 0pt, minimum size = 4pt] at (1.5, -0.25) {};
            \node[fill = red, circle, inner sep = 0pt, minimum size = 4pt] at (2, 0.65) {};
            \node[fill = red, circle, inner sep = 0pt, minimum size = 4pt] at (3, 1.25) {};
            
            \draw[red, thick, latex-] (0.5, -1.5) -- (0.5, 1.5);
            \draw[red, thick, latex-] (3, -1.5) -- (3, 1.25);
            \draw[red, thick, latex-] (1.5, -1.5) -- (1.5, -0.25);
            \draw[red, thick, latex-] (2, -1.5) -- (2, 0.65);

            \node[fill = blue, circle, inner sep = 0pt, minimum size = 4pt] (a) {};
            \node[fill = blue, circle, inner sep = 0pt, minimum size = 4pt, right = 2cm of a] (b) {};
            \node[fill = blue, circle, inner sep = 0pt, minimum size = 4pt, below right = 1cm of a] (c) {};
            \node[fill = blue, circle, inner sep = 0pt, minimum size = 4pt, right = 0.75cm of c] (d) {};
            \node[fill = blue, circle, inner sep = 0pt, minimum size = 4pt, above right = 1cm of a] (e) {};
            \node[fill = blue, circle, inner sep = 0pt, minimum size = 4pt, right = 1.75cm of e] (f) {};

            \draw[thick, blue] (a) -- (b);
            \draw[thick, blue] (c) -- (d);
            \draw[thick, blue] (e) -- (f);
        \end{tikzpicture}
        \caption{{\em There are three intersections between the horizontal line segments and the vertical rays.}}
        \label{fig:enter-label}
    \end{figure}

    \item We now revisit the original problem. Suppose that each vertical ray is bounded from below. In other words, each vertical ray has a top and bottom endpoint. Each of these vertical line segments can be specified by one $x$-coordinate and two $y$-coordinates $\{x_i, y_{i_1}, y_{i_2}\}$.
    
    Describe an $O(n(\log n)^3)$ algorithm that returns the number of intersections between line segments in $H$ and line segments in $V$.

    \begin{figure}[H]
        \centering
        \begin{tikzpicture}

            \node[fill = red, circle, inner sep = 0pt, minimum size = 4pt] at (0.5, -1.5) {};
            \node[fill = red, circle, inner sep = 0pt, minimum size = 4pt] at (0.5, 1.5) {};
            \node[fill = red, circle, inner sep = 0pt, minimum size = 4pt] at (1.5, -0.25) {};
            \node[fill = red, circle, inner sep = 0pt, minimum size = 4pt] at (1.5, -1.25) {};
            \node[fill = red, circle, inner sep = 0pt, minimum size = 4pt] at (2, 0.65) {};
            \node[fill = red, circle, inner sep = 0pt, minimum size = 4pt] at (2, -0.5) {};
            \node[fill = red, circle, inner sep = 0pt, minimum size = 4pt] at (3, 1.25) {};
            \node[fill = red, circle, inner sep = 0pt, minimum size = 4pt] at (3, 0) {};
            
            \draw[red, thick] (0.5, -1.5) -- (0.5, 1.5);
            \draw[red, thick] (3, 0) -- (3, 1.25);
            \draw[red, thick] (1.5, -1.25) -- (1.5, -0.25);
            \draw[red, thick] (2, -0.5) -- (2, 0.65);

            \node[fill = blue, circle, inner sep = 0pt, minimum size = 4pt] (a) {};
            \node[fill = blue, circle, inner sep = 0pt, minimum size = 4pt, right = 2cm of a] (b) {};
            \node[fill = blue, circle, inner sep = 0pt, minimum size = 4pt, below right = 1cm of a] (c) {};
            \node[fill = blue, circle, inner sep = 0pt, minimum size = 4pt, right = 0.75cm of c] (d) {};
            \node[fill = blue, circle, inner sep = 0pt, minimum size = 4pt, above right = 1cm of a] (e) {};
            \node[fill = blue, circle, inner sep = 0pt, minimum size = 4pt, right = 1.75cm of e] (f) {};

            \draw[thick, blue] (a) -- (b);
            \draw[thick, blue] (c) -- (d);
            \draw[thick, blue] (e) -- (f);
        \end{tikzpicture}
        \caption{{\em There are three intersections between the horizontal line segments and the vertical line segments.}}
        \label{fig:enter-label}
    \end{figure}

    {\bfseries Note.} {\em You can obtain an $O(n (\log n)^2)$ and $O(n \log n)$-time algorithm by doing a pre-processing step, either in part (b) or part (c).}
\end{enumerate}
\end{question}

\begin{rubric}
\begin{itemize}
    \item Your solution should clearly outline which subpart you're answering.

    \item If your solution is a simple modification of a previous solution, you do not need to restate the solution; you can simply refer to the previous part in your solution.

    \item As usual, you should argue the correctness of the algorithm and its time complexity.

    \item This task will form part of the portfolio.
    \item Ensure that your argument is clear and keep reworking your solutions until your lab demonstrator is happy with your work.
\end{itemize}
\end{rubric}

\begin{solution}
\begin{enumerate}[label = (\alph*)]
    \item Sort the vertical lines in increasing order of x-coordinate value.
    For each of the horizontal line segments, perform a binary search to find the first 
    vertical line for which the left end of the horizontal line segment lies to the right.
    Do the same for the right end of the line segment.
    By subtracting the indices, you get the number of times each horizontal line segments intersects a vertical line.
    Summing all these differences together gives the total number of intersections.

    \item We will solve this problem using a divide and conquer approach. Solution to part (a).

    \textbf{Divide}

    Sort both the horizontal line segments and vertical ray end-points in increasing order of y-coordinate.
    Divide both the $V$ and $H$ arrays based on whether each element is above or below the median element.
    We split $V$ into $V_B$ (below) and $V_A$ (above). Similarly we can split $H$ into $H_B$ (below) and $H_A$ (above).
    Note that $V_B + H_B = V_A + H_B = n/2$.

    We can divide the problem now into two subproblems of size $n/2$. Each problem returns an integer, being the number of intersections.
    \begin{itemize}
        \item $V_B$ and $H_B$
        \item $V_A$ and $H_A$
    \end{itemize}

    \textbf{Combine}

    The $(V_B, H_B)$ subproblem will account for the intersections of the vertical rays in $V_B$
    with the horizontal line segments in $H_B$.
    The $(V_A, H_A)$ subproblem will account for the intersections of the vertical rays in $V_A$
    with the horizontal line segment sin $H_A$.
    There is still one set of intersections which need to be accounted for, which are the intersections 
    of the vertical rays in $V_A$ with the horizontal line segments below in $H_B$.

    Since all the $V_A$ rays are guaranteed to start above all the line segments in $H_B$, 
    they may as well be treated as lines extending in both directions up and down. 
    We can use the algorithm describe in part (a) to solve this step in $O(n\log n)$ time.

    \textbf{Base Case}
    The base case occurs when there is only one entity in total. 
    In both cases, there are zero intersections.

    \textbf{Time Complexity}

    blah blah master theorem


    \item 
    Use part (b) to find the intersection of the horizontal lines in $H$ with downward rays originating from the top endpoints of all the vertical line segments in $V$.
    Do the same thing but with downward rays originating from the bottom endpoints of the vertical line segments in $V$. Subtract these two results.
    The first application of the algorithm overcounts the number of intersections, FINISH THIS LATER!!

\end{enumerate}
\end{solution}
\end{document}