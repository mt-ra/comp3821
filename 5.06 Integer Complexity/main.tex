\documentclass{article}

% Remember to also load the algostyle.sty file into your project.
\usepackage{algostyle}

% Insert new packages here.

\begin{document}
\begin{question}
Given a positive integer $n$, the {\em complexity} of $n$ is the minimum number of ones that can be used to represent $n$, using only the operations of addition and multiplication, as well as parenthesisation.

For example, we have the following representations:
\begin{align*}
    6 &= (1 + 1 + 1) \times (1 + 1). \\
    8 &= (1 + 1) \times (1 + 1) \times (1 + 1). \\
    9 &= (1 + 1 + 1) \times (1 + 1 + 1). \\
    12 &= (1 + 1 + 1 + 1) \times (1 + 1 + 1). \\
    19 &= (1 + 1 + 1) \times (1 + 1 + 1) \times (1 + 1) + 1.
\end{align*}

The first twenty entries are given for you.
\begin{center}
    \begin{tabular}{c|cccccccccccccccccccc}
        $n$ & 1 & 2 & 3 & 4 & 5 & 6 & 7 & 8 & 9 & 10 & 11 & 12 & 13 & 14 & 15 & 16 & 17 & 18 & 19 & 20 \\
        \hline & 1 & 2 & 3 & 4 & 5 & 5 & 6 & 6 & 6 & 7 & 8 & 7 & 8 & 8 & 8 & 8 & 9 & 8 & 9 & 9
    \end{tabular}
\end{center}

\begin{enumerate}[label = (\alph*)]
    \item Show that every positive integer can be represented by a string of ones, along with addition, multiplication, and parenthesisation operations; that is, the complexity of $n$ is always finite.

    \item Given a positive integer $n$, describe an $O(n^2)$ algorithm to compute the minimum number of one's (1's) using only the operations of addition and multiplication, as well as parentheses, whose expression equals $n$.
\end{enumerate}

{\bfseries Note.} {\em This is also known as the Mahler-Popken complexity. Here is the \href{https://oeis.org/A005245}{OEIS} entry.}
\end{question}

\begin{solution}
\begin{enumerate}[label = (\alph*)]
    \item Solution to part (a) goes here...

    \item Solution to part (b) goes here...
\end{enumerate}
\end{solution}
\end{document}