\documentclass{article}

% Remember to also load the algostyle.sty file into your project.
\usepackage{algostyle}

% Insert new packages here.

\begin{document}
\begin{question}
Given a positive integer $n$, the {\em complexity} of $n$ is the minimum number of ones that can be used to represent $n$, using only the operations of addition and multiplication, as well as parenthesisation.

For example, we have the following representations:
\begin{align*}
    6 &= (1 + 1 + 1) \times (1 + 1). \\
    8 &= (1 + 1) \times (1 + 1) \times (1 + 1). \\
    9 &= (1 + 1 + 1) \times (1 + 1 + 1). \\
    12 &= (1 + 1 + 1 + 1) \times (1 + 1 + 1). \\
    19 &= (1 + 1 + 1) \times (1 + 1 + 1) \times (1 + 1) + 1.
\end{align*}

The first twenty entries are given for you.
\begin{center}
    \begin{tabular}{c|cccccccccccccccccccc}
        $n$ & 1 & 2 & 3 & 4 & 5 & 6 & 7 & 8 & 9 & 10 & 11 & 12 & 13 & 14 & 15 & 16 & 17 & 18 & 19 & 20 \\
        \hline & 1 & 2 & 3 & 4 & 5 & 5 & 6 & 6 & 6 & 7 & 8 & 7 & 8 & 8 & 8 & 8 & 9 & 8 & 9 & 9
    \end{tabular}
\end{center}

\begin{enumerate}[label = (\alph*)]
    \item Show that every positive integer can be represented by a string of ones, along with addition, multiplication, and parenthesisation operations; that is, the complexity of $n$ is always finite.

    \item Given a positive integer $n$, describe an $O(n^2)$ algorithm to compute the minimum number of one's (1's) using only the operations of addition and multiplication, as well as parentheses, whose expression equals $n$.
\end{enumerate}

{\bfseries Note.} {\em This is also known as the Mahler-Popken complexity. Here is the \href{https://oeis.org/A005245}{OEIS} entry.}
\end{question}

\begin{solution}
\begin{enumerate}[label = (\alph*)]
    \item Every positive integer $n$ can be represented as the sum of $n$ ones.

    \item 

    \textbf{Subproblems}

For $1\leq i\leq n$, let $P(i)$ be the problem of determining the 
complexity of $i$.

\textbf{Recurrence}

Either $i$ is a product, or a sum.

Let $\mathrm{prod}(i)$ be the minimum complexity of $i$ if it was a product.
Let $\mathrm{sum}(i)$ be the minimum complexity of $i$ if it was a sum.

$$
\mathrm{sum}(i) = 1 + \mathrm{opt}(i-1).
$$

$$
\mathrm{prod}(i) = \min_{2\leq j\leq \sqrt{i}}\begin{cases}
    \mathrm{opt}(i) + \mathrm{opt}(i/j) &\text{if $j|i$}\\
    \infty &\text{otherwise}
\end{cases}.
$$

$$
\mathrm{opt}(i)=\min(\mathrm{sum}(i), \mathrm{prod}(i)).
$$

\textbf{Base Cases}

$$\mathrm{opt}(1)=1.$$

\textbf{Order of Computation}

Compute in increasing order of $i$.

\pagebreak

\textbf{Overall Answer}

$$\mathrm{opt}(n).$$

\textbf{Time Complexity}

There are $O(n)$ subproblems, and computing each one is either $O(1)$, or involves scanning through $O(n)$
candidate factors to find the optimal factorisation.


\end{enumerate}
\end{solution}
\end{document}