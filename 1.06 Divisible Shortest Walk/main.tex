\documentclass{article}

% Remember to also load the algostyle.sty file into your project.
\usepackage{algostyle}

% Insert new packages here.

\begin{document}
\begin{question}
Let $G = (V, E, w)$ be a directed and weighted graph with edge weights $w(e)$, which can be negative or positive or zero, for each edge $e \in E$. You may assume that $G$ has no {\em negatively weighted} cycles. Additionally, each vertex is coloured either {\em red} or {\em blue}. Recall that a {\em walk} is a sequence of edges joining a sequence of vertices; there are no restrictions on the edges or vertices (i.e. they may repeat or not repeat).

Given an integer $k \geq 2$, describe an $O(kn(km + n))$ algorithm that returns a walk with the smallest weight, satisfying the following conditions:
\begin{itemize}
    \item The walk starts and ends on a vertex coloured red, and
    \item The number of vertices coloured blue in the walk is divisible by $k$.
\end{itemize}

{\bfseries Note.} {\em For a directed and weighted graph $G = (V, E, w)$ where the edge weights can be negative, Bellman-Ford will compute the shortest distance from a vertex $u$ to every vertex in time $O(\lvert V \rvert \cdot \lvert E \rvert)$. We will see how it works later but for now, you may use it as a black box.}

{\bfseries Hint.} {\em Try constructing your edges in layers modulo $k$. You should have $km + 2n$ edges and $kn + 2$ vertices in your new graph.}
\end{question}

\begin{solution}
    We can remodel the question with an auxilliary graph $A$, which consists of $k$ layers (zero-indexed), each containing
    almost identical copies of $G$, except for the fact that all out-edges from blue vertices, 
    instead of pointing to their neighbour in the current layer, they point to the equivalent of their neighbour 
    one layer above, with blue vertices in the $(k-1)$th layer having out-edges pointing towards vertices in the $0$th layer.
    This index of the layer you are currently in represents the number of blue vertices you have crossed so far modulo $k$.

    We add a special \textsc{End} vertex, with $0$-weighted inedges from all red vertices in layer $0$.
    We should also have a special \textsc{Begin} vertex, which has $0$-weighted out-edges pointing to every red vertex in layer $0$.
    Applying Bellman-Ford's algorithm on $A$ from the source vertex \textsc{Begin}, and finding the shortest path ending at vertex \textsc{End},
    will give us our desired path.

    The algorithm is correct because the layer you are currently in represents the number of blue vertices you have crossed so far modulo $k$,
    and so you can only reach the \textsc{End} vertex if your current path has a number of blue vertices divisible by $k$.

    Without considering the extra nodes and vertices added due to the \textsc{Begin} and \textsc{End} vertices, 
    there are $km$ edges and $kn$ vertices, as there are essentially $k$ copies of $G$ except that out-edges from 
    blue vertices have been redirected. The addition the \textsc{Begin} and \textsc{End} introduces at most $n$ new edges and exactly $1$ new vertex each.
    Therefore the auxilliary graph $A$ has at most $km + 2n$ edges and exactly $kn + 2$ vertices. Therefore the time complexity 
    of this algorithm is 
    $$O(\textsc{numVertices}(A)\cdot \textsc{numEdges}(A))=O((kn+2)(km + 2n)) = O(kn(km+n)).$$
\end{solution}
\end{document}