\documentclass{article}

% Remember to also load the algostyle.sty file into your project.
\usepackage{algostyle}

% Insert new packages here.

\begin{document}
\begin{question}
We are given a directed and weighted graph $G = (V, E, w)$, where $w(e) > 0$ for each $e \in E$. Additionally, each edge is coloured either red or blue.

\begin{enumerate}[label = (\alph*)]
    \item Let $\compproblem{onlyRedEdges}(u, v, \ell)$ denote the weight of the shortest path from $u$ to $v$ that uses {\em only} red edges and have length at most $\ell$. For a fixed pair of points $u$ and $v$, describe an $O(n^3 \log k)$ algorithm to compute $\compproblem{onlyRedEdges}(u, v, k - 1)$.

    {\bfseries Hint.} {\em You should only need $O(n^2 \log k)$ subproblems.}

    {\bfseries Hint 2.} {\em Alternatively... you can just apply Bellman-Ford.}

    \item Hence, describe an $O(n^3 \log k)$ algorithm that returns a {\em closed walk} with the smallest weight that contains {\em at least} one blue edge and does not contain $k$ consecutive red edges.

    A {\em closed walk} is a walk that starts and ends at the same vertex.
\end{enumerate}
\end{question}

\begin{solution}
\begin{enumerate}[label = (\alph*)]
    \item 

    \textbf{Subproblems}

    For $u, v\in V$ and $1\leq i \leq k$, let $P(u, v, i)$ be the problem of determining $\mathrm{opt}(u, v, i)$, the shortest 
    path from $u$ to $v$ that uses only red edges, and has a length at most $i$.

    \textbf{Recurrence}

    If $i$ is even then
    $$
    \mathrm{opt}(u, v, i) = \min_{w \in V}(\mathrm{opt}(u, w, i/2) + \mathrm{opt}(w, v, i/2)).
    $$

    If $i$ is odd then
    $$
    \mathrm{opt}(u, v, i) = \min_{w \in V}(\mathrm{opt}(u, w, i - 1) + \mathrm{opt}(w, v, 1)).
    $$

    Note that $w$ can be equal to $u$ or $v$, which is the same as allowing the choice to not 
    choose any intermediate vertex at all.

    \textbf{Base Cases}

    If $u = v$ then $\mathrm{opt}(u, v, i)=0$ for all $i$. 
    
    If $(u,v)\in E$, then $\mathrm{opt}(u, v, 1)=w((u, v))$. Otherwise $\mathrm{opt}(u, v, 1) = \infty$.

    \textbf{Order of Computation}

    Starting from $i = 1$, we can always reach $k-1$ using a sequence of doubling $i$, or adding 1 to $i$. 
    The shortest such sequence of operations can be found by looking at the binary representation of $k-1$. 
    Starting from right to left, if there is a $0$ then multiply $i$ by 2, and compute $P(u,v,i)$ for all $u,v\in V$.
    If there is a $1$, then add $1$ to $i$ and compute $P(u,v,i)$, and then muliply $i$ by 2 and compute $P(u,v,i)$, for all $u,v\in V$.

    Continue this until $P(u, v, k-1)$ has been computed for all $u,v\in V$.


    \textbf{Overall Answer}
    $$\mathrm{opt}(u, v, k-1).$$

    \textbf{Time Complexity}

    The sequence of operations that we perform on $i$ as in increases towards $k-1$
    has a size which is on the order of the number of digits in the binary representation 
    of $k-1$, which is $O(\log k)$. Therefore there are $O(n^2 \log k)$ subproblems.
    Each takes $O(n)$ time to compute, giving a total of $O(n^3 \log k)$.

    \pagebreak
    \item 
    After running the $O(n^3 \log k)$ DP to compute $\textsc{onlyRedEdges}(u, v, k-1)$, we can query the result of this function for any pair $u,v\in V$ in constant time, since the result is memoized.

    We will construct an auxilliary graph with two layers: a red layer and a blue layer.
    Each layer contains $n$ vertices, each vertex corresponding to one of the vertices in $G$.
    The layer that a visitor is in determines the colour of the edge that the visitor has last crossed.

    If $(u,v)$ is an blue edge in $G$, then let there be an edge from the 
    $u_{\text{blue}}$ to $v_{\text{blue}}$ and an edge from 
    $u_{\text{red}}$ to $v_{\text{blue}}$, both edges
    having a weight of $w(u,v)$.
    This represents the fact that after a visitor has crossed a blue edge, 
    they can cross another blue edge. 
    Also if a visitor has just crossed a red edge, then they can also take a blue edge.

    For all pairs of vertices $(u,v)$ in $G$ such that $\textsc{onlyRedEdges}(u, v, k-1)$ is not infinite,
    let there be an edge from $u_{\text{blue}}$ to $v_{\text{red}}$ with a 
    weight of $\textsc{onlyRedEdges}(u, v, k-1)$. 
    This represents the fact that only after crossing a blue edge, can you take a path 
    of red edges.

    Constructing this graph takes $O(n^2)$ time, as each pair of vertices must be considered.
    Perform Floyd-Warshall's algorithm on this auxilliary graph, taking $O(n^3)$ time.
    This Floyd-Warshall's algorithm must be modified such that the initial distances from each vertices 
    to themself is infinity. This ensures that only paths of non-zero length are considered.

    Out of all the vertices $u_{\text{red}}$ in the red layer, find the one such that the shortest path 
    on the auxilliary graph from $u_{\text{red}}$ to itself. This takes $O(n)$ time.

\end{enumerate}
\end{solution}
\end{document}