\documentclass{article}

% Remember to also load the algostyle.sty file into your project.
\usepackage{algostyle}

% Insert new packages here.

\begin{document}
\begin{question}
A battle takes place between $n$ heroes and $m$ monsters. Let $h_j$ denote the starting amount of health points of monster $j$, where $1 \leq j \leq m$. During one round of the battle, each hero attacks a monster of their own choice, causing one damage to the monster's health points. If a monster's number of health points reduces to zero at any time, they are killed. At the end of each round, all monsters still alive will gain $k$ health points; therefore, a monster can exceed $h_j$ health points. The monsters will not attack any of the heroes.

The heroes win the battle if they manage to kill all of the monsters.

\begin{enumerate}[label = (\alph*)]
    \item Suppose that $k < n$. Describe a strategy that chooses which monster(s) to attack in each round, in order to guarantee victory for the heroes in the fewest number of rounds as possible.

    \item Prove that your algorithm guarantees that the heroes win.

    \item Briefly explain why your strategy guarantees that the heroes win in the fewest number of rounds possible.

    \item Now, suppose that $k \geq n$. Under what circumstance(s) can the heroes win the battle? Briefly explain why.
\end{enumerate}
\end{question}

\begin{solution}
\begin{enumerate}[label = (\alph*)]
    \item Each hero on their turn attacks the alive monster with the least number of health points, 
    breaking ties arbitrarily.

    \item When a monster starts getting attacked by a hero, all heroes will participate.
    This is because heroes will attack the monster with the least number of health points.
    After a monster with the least number of health points gets attacked, 
    they retain their position as the monster with the least number of health points,
    and so all heroes will target this monster.
    Therefore for each turn that the monster is still alive, 
    the monster will lose $n$ health points, and regenerate $k$ which is less than $n$.
    Thus for any monster, once it starts being targeted by the heroes, it is guaranteed to die eventually.
    All monsters at some point in the game will be ``the alive monster with the least number of health points'', 
    therefore all monsters will die eventually.

    \item We want to show that the total damage that has to be dealt by the heroes to kill all the monsters,
    using this greedy strategy, is minimised. 
    Since a constant amount of damage is dealt each round, this must mean that our greedy strategy also minimises the number of rounds.
    By using strong induction we can show that for all integers $1 \leq i \leq m$, 
    that the total damage it takes to kill $i$ monsters is minimised in the greedy strategy.

    It is obvious that the way to kill $1$ monster with the least amount of total damage, 
    is by focusing all the heroes' attacks on the monster with least health points.

    For the inductive step, assume that the statement is true for all integers $i$ from $1$ to $j$.
    We want to show that it is also true for $i=j+1$. 
    Obviously, if we want to kill $i+1$ monsters in the fastest possible way, 
    we would only focus on those $i+1$ monsters, and not attack any other monsters.
    Therefore the total damage that needs to be done is equal to the sum 
    of the initial hitpoints of the $i+1$ monsters, 
    plus the amount of health regenerated by these $i+1$ monsters as a whole during the course of this massacre.
    Our goal now is to show that the amount of health regenerated is least in the greedy strategy.

    The amount of health regenerated in a round is equal to $k$ times the number of monsters alive at the end of that round.
    If you compare the times of monster deaths from the greedy strategy compared to any other strategy, 
    by the inductive hypothesis, all deaths in the greedy strategy occur sooner.
    Therefore the number of monsters alive at the end of a round will always be less 
    in the greedy strategy. Therefore by induction, for all integers from $1$ to $m$,
    using the greedy strategy results in the least number of damage required to kill that many number of monsters.
    Therefore our greedy strategy minimises the number of rounds.


    \item If the total number of initial hitpoints is less than or equal to $n$, 
    then the heroes can win immediately on the first turn. 
    However if there is at least one monster remaining, then the heroes will never win.
    This is because the monsters' hitpoints will go up faster than or equal to the rate the heroes can damage the monster.
\end{enumerate}
\end{solution}
\end{document}