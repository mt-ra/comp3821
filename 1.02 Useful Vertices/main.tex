\documentclass{article}

% Remember to also load the algostyle.sty file into your project.
\usepackage{algostyle}

% Insert new packages here.

% Commands
\DeclareMathOperator{\pred}{pred}

\begin{document}
\begin{question}
Let $G = (V, E, w)$ be a directed and weighted graph with positive edge weights $w(e) > 0$ for each edge $e \in E$. For a pair of vertices $u, v \in V$, there may be {\em multiple} shortest paths from $u$ to $v$. Let $\Pi_{u, v}$ denote all such paths. In other words, for a pair of vertices $u$ and $v$, $\Pi_{u, v}$ is the set of all shortest paths from $u$ to $v$.

A vertex $x$ is called {\em useful} if $x$ lies on {\em any} path in $\Pi_{u, v}$. Given the graph $G$ and a pair of vertices $u, v \in V$, describe an $O(m \log n)$ algorithm to return all useful vertices.
\end{question}

\begin{solution}
Applying Dijkstra's algorithm on $G$ with source $u$ will associate each vertex $v\in V$ with the 
set of possible predecessors of $v$ in the shortest paths from $u$ to $v$. In other words, 
Dijkstra's algorithm gives us the function $\pred:V\to \mathcal{P}(V)$, where $\pred$ is defined such that 
for all $v\in V$, $\pred(v) = \{x\in V : (\exists p\in\Pi_{u,v}\quad \text{such that } (x,v)=p.\textit{back})\}$.
We can return a reference to a predecessor set of a vertex from this function in $O(1)$ time.

Let $B=\{(a, b)\in V\times V : b\in\pred(a)\}$, representing the set of all edges which exist in shortest paths, but reversed. Let $A=(V, B)$ be a directed and unweighted graph. Perform 
a BFS on $A$ from the starting vertex $v$, using the $\pred$ function as an adjacency list.
Insert all processed vertices into a set and return the set.

This algorithm is correct because if $a$ is the predecessor of $v$ in a shortest path from $u$ to $v$, 
then all useful vertices from $u$ to $a$ are also useful vertices from $u$ to $v$.

The Dijkstra's algorithm step is $O(m\log n)$. The BFS step will take $O(m)$ time. Therefore the whole algorithm together is $O(m\log n + m) = O(m\log n)$.





\end{solution}
\end{document}