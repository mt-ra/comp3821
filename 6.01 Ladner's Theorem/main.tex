\documentclass{article}

% Remember to also load the algostyle.sty file into your project.
\usepackage{algostyle}

% Insert new packages here.

\begin{document}
\begin{question}
\begin{enumerate}[label = (\alph*)]
    \item Suppose that $\mathsf P = \mathsf{NP}$. Prove that $\mathsf{NP} \setminus \left(\mathsf P \cup \mathsf{NP\text{-}C}\right) = \emptyset$.

    \item Suppose now that $\mathsf{NP} \setminus \left(\mathsf P \cup \mathsf{NP\text{-}C}\right) = \emptyset$. Prove that $\mathsf P = \mathsf{NP}$.
\end{enumerate}

{\bfseries Note.} {\em This is the corollary of Ladner's theorem.}
\end{question}

\begin{solution}
\begin{enumerate}[label = (\alph*)]
    \item 
    \begin{align*}
    \mathsf{NP} \setminus \left(\mathsf P \cup \mathsf{NP\text{-}C}\right) &= \mathsf{NP} \setminus (\mathsf{NP} \cup \mathsf{NP\text{-}C})\\
    &= \mathsf{NP} \setminus (\mathsf{NP} \cup (\mathsf{NP}\cap \mathsf{NP\text{-}H}))\\
    &= \mathsf{NP} \setminus \mathsf{NP}\\
    &= \emptyset
    \end{align*}

    \item 
    Ladner's theorem states that if $\mathsf{P}\neq \mathsf{NP}$, then there exists a language in $\mathsf{NP}$ that 
    is neither in $\mathsf{P}$, nor $\mathsf{NP\text{-}C}$.
    This means that if $\mathsf{P}\neq \mathsf{NP}$ then $\mathsf{NP} \setminus \left(\mathsf P \cup \mathsf{NP\text{-}C}\right)$ is non-empty. Taking the contrapositive gives us that if $\mathsf{NP} \setminus \left(\mathsf P \cup \mathsf{NP\text{-}C}\right)=\emptyset$ then $\mathsf{P} = \mathsf{NP}$ as required.
 
\end{enumerate}
\end{solution}
\end{document}