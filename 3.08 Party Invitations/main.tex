\documentclass{article}

% Remember to also load the algostyle.sty file into your project.
\usepackage{algostyle}

% Insert new packages here.

\begin{document}
\begin{question}
Alice wants to throw a party and is deciding who to invite. She has $n$ people to choose from, and she has created a list consisting of pairs of people who know each other. She wants to invite as many people as possible, subject to  the following two constraints.
\begin{itemize}
    \item Each person should have at least five other people whom they know.
    \item Each person should also have at least five other people whom they do not know.
\end{itemize}

Formally, you are given the list of $n$ people and the list of all pairs of people who {\em mutually} know each other; that is, you may assume that if $X$ knows $Y$, $Y$ also knows $X$. To make things simpler, we can model the problem as a graph. Each person is a vertex, and an {\em undirected} edge connects two people $X, Y$ if and only if $X$ and $Y$ mutually know each other. Therefore, the input can be represented by a graph $G = (V, E)$ and an adjacency matrix $A[1..n, 1..n]$.

\begin{enumerate}[label = (\alph*)]
    \item Describe an $O(n^3)$ algorithm to return a subset of maximum cardinality of people satisfying the two constraints.

    \item Argue that your algorithm produces an optimal subset.

    \item Prove that the optimal subset is {\em unique}.

    {\bfseries Hint.} {\em Union-closed families have a unique ``largest element''.}
\end{enumerate}
\end{question}

\begin{solution}
\begin{enumerate}[label = (\alph*)]
    \item The basic idea is start with all vertices, and to keep removing vertices which violate the requirements.
    First, for each vertex, find how many people they know, and how many people they do not know, and put this information in an array.
    We then begin a loop where we repeatedly remove vertices which have less than 5 people they know or less than 5 people they do not know.

    For each iteration, scan through all vertices until you find one that has less than 5 people they don't know, or less than 5 people they know.
    Mark this vertex as ``removed''. Then scan through all vertices. 
    For each vertex, we want to update the array containing information about how many people they know and do not know
    For each vertex, if is not removed, and it adjacent to the removed vertex, then decrement the number of people that person knows.
    Otherwise, decrement the number of people that person does not know.

    Continue this until there aren't any unremoved vertices that violate the requirements.
    The vertices remaining should form an optimal subset of people who can be invited.
    
    Finding the array of the number of people that a person knows and does not know takes $O(n^2)$ time.
    At most $n$ vertices will be removed, and at each stage, removing an element requires scanning through all $n$ vertices.
    Therefore the algorithm is $O(n^2)$.

    \item Every person who is removed from the selection by this algorithm must be removed.
    Suppose that person $u$ does not satisfy the requirements. 
    Removing any combination of other people will not increase the number of people that $u$ knows, or does not know.
    Therefore there does not exist a valid selection where $u$ is in it. 
    Therefore all removals made by this algorithm are necessary, and there are no unnecessary removals.
    This this is the optimal subset.

    \item The union of two valid selections is also a valid selection. 
    This is because adding people to an already valid selection does not 
    decrease the number of people each person from the original selection knows, or does not know.
    Therefore the family of valid selections is closed under union, and thus there is a unique largest element.
\end{enumerate}
\end{solution}
\end{document}