\documentclass{article}

% Remember to also load the algostyle.sty file into your project.
\usepackage{algostyle}

% Insert new packages here.

\begin{document}
\begin{question}
You are given a long piece of stick of length $L$. You want to cut the stick into {\em exactly} $n$ places along its length, where the $i$th place to cut occurs at position $A[i]$. Since sticks of larger size require more power, cutting a stick of length $X$ requires $X$ units of work.

\begin{enumerate}[label = (\alph*)]
    \item Given the stick of length $L$ and the positions of the $n$ places to cut $A[1..n]$, describe an $O(n^3)$ algorithm to compute the minimum number of units of work to cut the stick into $n$ pieces.

    \item Describe an $O(n^2)$ algorithm to compute the minimum number of units of work to cut the stick into $n$ pieces.

    {\bfseries Hint.} {\em What can we optimise here?}
\end{enumerate}
\end{question}

\begin{solution}
\begin{enumerate}[label = (\alph*)]
    \item 

    \textbf{Subproblems}

    Define some dummy array elements $A[0] = 0$ and $A[n+1] = L$.
    For $0\leq i \leq j \leq n+1$, let $P(i, j)$ be the problem of determining the minimum number of units of work required 
    to cut the stick which begins at $A[i]$ and ends at $A[j]$, cutting at the positions $A[(i+1)..(j-1)]$.

    \textbf{Recurrence}
    $$
    \mathrm{opt}(i,j)=A[j] - A[i] + \min_{i<k<j}(\mathrm{opt}(i,k) + \mathrm{opt}(k,j))
    $$

    \textbf{Base Cases}

    For all $i$, $\mathrm{opt}(i, i+1) = 0$.

    \textbf{Order of Computation}

    Compute the subproblems $P(i,j)$ in order of increasing difference $j-i$.

    \textbf{Overall Answer}
    $$ \mathrm{opt}(0,n+1).$$

    \textbf{Time Complexity}

    There are $O(n^2)$ subproblems, each taking $O(n)$ time to compute, for a total of $O(n^3)$.

    \item 

    \textbf{Subproblems}

    Define some dummy array elements $A[0] = 0$ and $A[n+1] = L$.
    For $0\leq i \leq j \leq n+1$, let $P(i,j)$ be the problem of determining the 
    minimum number of units of work $\mathrm{opt}(i,j)$ required to cut the stick 
    beginning at $A[i]$ and ending at $A[j]$, as well as determining $\mathrm{spl}(i,j)$
    which is the optimal index $i < k < j$ at which to cut the stick to ensure the least 
    amount of work.

    \textbf{Recurrence}

    $$
    \mathrm{opt}(i,j)=A[j] - A[i] + \min_{\mathrm{spl}(i,j-1)<k<\mathrm{spl}(i+1,j)}(\mathrm{opt}(i,k) + \mathrm{opt}(k,j))
    $$

    $$
    \mathrm{spl}(i,j)= \mathrm{arg}\min_{\mathrm{spl}(i,j-1)<k<\mathrm{spl}(i+1,j)}(\mathrm{opt}(i,k) + \mathrm{opt}(k,j))
    $$


    \textbf{Base Cases}

    For all $i$, $\mathrm{opt}(i, i+1) = 0$ and $\mathrm{spl}(i, i+2) = i+1$.

    \textbf{Order of Computation}

    Compute the subproblems $P(i,j)$ in increasing difference $j-1$.

    \textbf{Overall Answer}
    $$ \mathrm{opt}(0,n+1).$$

    \textbf{Time Complexity}

    Since the ranges $(\mathrm{spl}(i,j-1), \mathrm{spl}(i+1,j)), (\mathrm{spl}(i+1, j), \mathrm{spl}(i+2, j+1), \dots$ are non-intersecting, scanning through all possible split locations 
    for a certain difference $j-i$ takes $O(n)$ time in total. 
    Since there are $O(n)$ possible differences $j-i$, the full algorithm takes $O(n^2)$ time.

\end{enumerate}
\end{solution}
\end{document}