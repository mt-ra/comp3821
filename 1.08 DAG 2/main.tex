\documentclass{article}

% Remember to also load the algostyle.sty file into your project.
\usepackage{algostyle}

% Insert new packages here.

\begin{document}
\begin{question}
Let $G = (V, E)$ be a directed and acyclic graph.

\begin{enumerate}[label = (\alph*)]
    \item Show that $G$ has {\em at most} one Hamiltonian path. Recall that a Hamiltonian path in $G$ is a path that visits every vertex exactly once.

    \item Prove that $G$ has a {\em unique} topological order if and only if $G$ has a Hamiltonian path.

    {\bfseries Hint.} {\em First, show that if $G$ has a Hamiltonian path $P$, then the order of the vertices in $P$ is, in fact, a topological order. Then, show that if $G$ has no Hamiltonian paths, then $G$ must have at least two distinct topological orders.}

    \item Prove that $G$ has a {\em unique} topological order if and only if, for every pair of vertices $u, v \in V$, either there exist a path from $u$ to $v$ or there exist a path from $v$ to $u$. In other words, every pair of vertices are {\em comparable}.

    \item Hence, describe an $O(m + n)$ algorithm to decide if a directed and acyclic graph $G$ has a Hamiltonian path. In other words, the $\compproblem{HamPath}$ problem is not a {\em hard} problem\footnote{The technical term for {\em hard} problem is $\mathsf{NP}$-complete, but we will see this towards the end of the course.} if $G$ is a DAG.
\end{enumerate}
\end{question}

\begin{solution}
\begin{enumerate}[label = (\alph*)]
    \item % fix to remove any reference to topological order
    Let $f:V\to \{1, \dots, n\}$ be the topological order of $G$. 
    In any path on a DAG, the topological order of each successive vertex must increase by definition.
    Suppose that $G$ had at least two Hamiltonian paths, two of them being $P_1$ and $P_2$, and let the first point of difference be $v_1$ in $P_1$ and $v_2$ in $P_2$ where $v_1\neq v_2$.
    Without loss of generality, let $f(v_1) < f(v_2)$. Since $P_1$ and $P_2$ have been identical up until this first point of difference, 
    $v_1$ has not yet appeared in $P_2$. Additionally, $v_1$ cannot appear in $P_2$ at any point after this, since topological order can only increase along a path. 
    This is a contradiction as Hamiltonian paths must contain all vertices.
    \item 
    If $G$ has a Hamiltonian path $P$, then the order of the vertices in $P$ is the topological order. 
    Since there are $n$ vertices in the Hamiltonian path, and the topological order along any path must increase, 
    the topological order from the start to the finish of the Hamiltonian path must increase in a counting fashion.
    There is only one ordering that satisfies this requirement, therefore $G$ has a unique topological order if it has a Hamiltonian path.

    If all edgespairs of vertices with consecutive topological order existed, then $G$ would have a Hamiltonian path.
    If for all integers $k$ in the range $\{1,\dots, n-1\}$, there existed a directed edge from $f^{-1}(k)$ to $f^{-1}(k + 1)$, then $G$ would have a Hamiltonian path.
    In other words, if for all pairs of vertices $(u,v)$, such that $u$ and $v$ were consecutive in the topological order, were elements of $E$, then $G$ would have a Hamiltonian path.
    Therefore if $G$ does not have a Hamiltonian path, then this means that there exists consecutive numbers $a, a + 1$ which correspond to the vertices
     $u=f^{-1}(a), v=f^{-1}(a+1)$, such that $(u, v)\notin E$.
    Swapping the order of $u$ and $v$ still results in a valid topological order, because all vertices which point to $v$ had a order of at most $f(u) - 1$ due to the fact
    that $u$ does not point to $v$. Similarly out-edges of $u$ have a order of at least $f(v) + 1$, and therefore swapping the order is valid.
    Therefore if $G$ has no Hamiltonian path, then $G$ has at least two distinct topological orders.

    Therefore $G$ has a unique topological ordering if and only if $G$ has a Hamiltonian path.
    \pagebreak
    \item 
    If $G$ has a unique topological order, then $G$ has a Hamiltonian path. Therefore for any pair of vertices $u,v\in V$, 
    you can find a path either from $u$ to $v$ or from $v$ to $u$ by choosing the vertex that comes 
    earlier in the Hamiltonian path, and following the path until you reach the latter vertex.

    If for every pair of vertices $u,v\in V$ there exists either a path from $u$ to $v$ or from $v$ to $u$, 
    then we can use the fact that the graph is acyclic to deduce that their exists a Hamiltonian path using construction.
    
    
    For any path $p$ and a vertex $v$ not in $p$, we can split $p$ into two subpaths $p_1$ and $p_2$ such that 
    all vertices in $p_1$ have a path to $v$ and all vertices in $p_2$ have a path from $v$. In order to prove this 
    by contradiction, suppose that there existed a pair of vertices in the path $a, b\in V$ such that $a$ occurs earlier in the path 
    than $b$, and such that $a$ does not have a path to $v$, and $v$ does not have a path to $b$.
    Then there must be a path from $v$ to $a$ and a path from $b$ to $v$. However, this would form a cycle since there is also a path from $a$ to $b$.
    Therefore we can find the subpaths $p_1$, $p_2$. 
    
    We can then extend $p_1$ forwards until it reaches $v$, since $p_1.\textit{back}$ has 
    a path to $v$. Similarly we can extend $p_2$ backwards until it reaches $v$, since there is a path from $v$ to $p_2.\textit{front}$. 
    We also know that at no point during our extensions will we come across a vertex already in the path, as this would imply there is a cycle.
    Therefore for any path $p$ such that there exists vertices $v\in V$ that are not in the path, we can extend the path by at least $1$.
    Therefore by repeatedly performing this method while there are remaining vertices not in the path, we will eventually obtain a path which contains all vertices, 
    which is a Hamiltonian path, which implies that $G$ has a unique topological order.

    \item 
    We can perform Kahn's algorithm, however if at any point there are two possible choices for the next vertex in the topological order,
    we are able to deduce that there is no unique topological order and hence no Hamiltonian path.

    Assume that our vertices are labelled from $0$ to $n-1$, and that the graph is given by an adjancency list representation.
    Start by constructing the array \textsc{NumParents}, where $\textsc{NumParents}[i]$ is the number of of vertices that have out-edges 
    which point towards vertex $i$. Filling this out will involve iterating through all vertices, and for each vertex $i$, increment $\textsc{NumParents}[c]$ for each of its children $c$.
    This process will take $O(m +n)$ time. We then perform a linear search to find the unique source in $O(n)$ time. 
    If there are multiple sources then there is no Hamiltonian path.

    Let $x$ be a mutable pointer to a vertex, initially pointing to the unique source. 
    We will then begin a loop, letting $x$ traverse the graph. With each iteration, 
    loop through each child $c$ of $x$, decrementing $\textsc{NumParents}[c]$, because what we are doing is essentially removing $x$ from the graph.
    Remember all children $c$ such that the decrement resulted in a zero value. If there was a single child such that this was the case, set $x$ to point to this vertex. 
    Otherwise if there were multiple, then this means there are multiple possible topological orders, implying that there is no Hamiltonian path.
    This stage of the algorithm is $O(m +n)$ because each vertex is visited at most once, and each edge is considered at most once.

\end{enumerate}
\end{solution}
\end{document}