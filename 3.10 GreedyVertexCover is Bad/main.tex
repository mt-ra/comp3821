\documentclass{article}

% Remember to also load the algostyle.sty file into your project.
\usepackage{algostyle}

% Insert new packages here.

\begin{document}
\begin{question}
Recall the $\compproblem{MinVertexCover}$ problem covered in lectures.
\begin{itemize}
    \item[] {\em Given an undirected graph $G = (V, E)$, return the size of the smallest subset $U \subseteq V$ of vertices such that every edge in $E$ is incident to at least one vertex in $U$.}
\end{itemize}

Consider the greedy heuristic: {\em consider the vertex with the largest degree, add that into the vertex cover, and remove the vertex from the graph and all incident edges.}

We will call this algorithm $\compproblem{GreedyVertexCover}$.

\begin{enumerate}[label = (\alph*)]
    \item Exhibit an instance of $G$ that shows that $\compproblem{GreedyVertexCover}$ is suboptimal.

    {\bfseries Note.} {\em You should provide an instance of a graph, what the greedy heuristic would choose, and what a more optimal solution be; the ``more optimal'' solution need not be the most optimal solution, it just needs to beat the greedy solution.}

    We will show that $\compproblem{GreedyVertexCover}$ can be made to perform arbitrarily badly; that is, it is not a constant-approximation algorithm. In particular, we will show that it is an $\Omega(\log n)$-approximation.

    \item We define the bipartite graph $G_n = (L \sqcup R, E)$, where $L$ is a set of $n$ vertices. We define $R$ in parts; for each $2 \leq i \leq n$, let $R_i$ denote a set of $\lfloor n/i \rfloor$ vertices, each with degree $i$. Then $R = R_1 \sqcup R_2 \sqcup \dots \sqcup R_n$. We define the edges such that all vertices of degree $i$ in $L$ are adjacent to distinct vertices in $R$.

    \begin{figure}[H]
        \centering
        \begin{tikzpicture}
            \node[circle, draw, fill = black!10] (a) {};
            \node[circle, draw, fill = black!10, right = 1cm of a] (b) {};
            \node[circle, draw, fill = black!10, right = 1cm of b] (c) {};
            \node[circle, draw, fill = black!10, right = 1cm of c] (d) {};
            \node[circle, draw, fill = black!10, right = 1cm of d] (e) {};
            \node[circle, draw, fill = black!10, right = 1cm of e] (f) {};

            \node[circle, draw = blue, fill = blue!10, below = 2cm of a] (a1) {};
            \node[circle, draw = red, fill = red!10, right = 1cm of a1] (b1) {};
            \node[circle, draw = green!60!black, fill = green!10, right = 1cm of b1] (c1) {};
            \node[circle, draw = purple, fill = purple!10, right = 1cm of c1] (d1) {};
            \node[circle, draw = purple, fill = purple!10, right = 1cm of d1] (d2) {};
            \node[circle, draw = yellow!60!black, fill = yellow!10, right = 1cm of d2] (e1) {};
            \node[circle, draw = yellow!60!black, fill = yellow!10, right = 1cm of e1] (e2) {};
            \node[circle, draw = yellow!60!black, fill = yellow!10, right = 1cm of e2] (e3) {};

            % draw the edges
            \draw[blue] (a1) -- (a);
            \draw[blue] (a1) -- (b);
            \draw[blue] (a1) -- (c);
            \draw[blue] (a1) -- (d);
            \draw[blue] (a1) -- (e);
            \draw[blue] (a1) -- (f);

            \draw[red] (b1) -- (a);
            \draw[red] (b1) -- (b);
            \draw[red] (b1) -- (c);
            \draw[red] (b1) -- (d);
            \draw[red] (b1) -- (e);

            \draw[green!60!black] (c1) -- (a);
            \draw[green!60!black] (c1) -- (b);
            \draw[green!60!black] (c1) -- (c);
            \draw[green!60!black] (c1) -- (d);

            \draw[red!60!black] (d1) -- (a);
            \draw[red!60!black] (d1) -- (b);
            \draw[red!60!black] (d1) -- (c);

            \draw[red!60!black] (d2) -- (d);
            \draw[red!60!black] (d2) -- (e);
            \draw[red!60!black] (d2) -- (f);

            \draw[yellow!60!black] (e1) -- (a);
            \draw[yellow!60!black] (e1) -- (b);
            \draw[yellow!60!black] (e2) -- (c);
            \draw[yellow!60!black] (e2) -- (d);
            \draw[yellow!60!black] (e3) -- (e);
            \draw[yellow!60!black] (e3) -- (f);
        \end{tikzpicture}
        \caption{{\em The graph $G_6$.}}
        \label{fig:enter-label}
    \end{figure}

    What does the greedy algorithm choose? What should the optimal vertex cover be?

    {\bfseries Hint.} {\em Firstly, figure out what the maximum degree of any vertex in $L$ must be and then use the greedy heuristic to decide what vertices the greedy algorithm picks.}

    \item Let $\lvert \compproblem{GreedyVertex} \rvert$ denote the size of the vertex cover chosen by the greedy algorithm, and let $\lvert \compproblem{opt}\rvert$ denote the size of the optimal vertex cover. Prove that \[\lvert \compproblem{GreedyVertex} \rvert \geq n(H_n - 2),\] where $H_n = \sum_{i = 1}^n 1/i$ is the $n$th Harmonic number.
    
    \item Hence, show that $\compproblem{GreedyVertexCover}$ is an $\Omega(\log n)$-approximation by proving that \[\frac{\lvert \compproblem{GreedyVertex} \rvert}{\lvert \compproblem{opt} \rvert} \geq H_n - 2.\]

    {\bfseries Note.} {\em $H_n = \log n + \Theta(1)$; therefore, proving the inequality shows the lower bound approximation.}
\end{enumerate}
\end{question}

\begin{solution}
\begin{enumerate}[label = (\alph*)]
    \item Solution to part (a) goes here.

    \item Solution to part (b) goes here.

    \item Solution to part (c) goes here.
    
    \item Solution to part (d) goes here.
\end{enumerate}
\end{solution}
\end{document}