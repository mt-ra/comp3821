\documentclass{article}

% Remember to also load the algostyle.sty file into your project.
\usepackage{algostyle}

% Insert new packages here.

\begin{document}
\begin{question}
Let $G = (V, E)$ be an undirected graph. A {\em bridge} edge is an edge whose deletion increases the number of connected components in $G$. Think of this as a cut vertex from Task 1.09 but on edges instead. A {\em bridgeless graph} is a graph with no bridge edges. Finally, a {\em strong orientation} assigns each edge of $G$ a direction such that $G$ becomes a strongly connected graph.

Robbins' theorem says that a connected graph $G$ has a strong orientation if and only if $G$ is bridgeless. Therefore, we will only look at bridgeless and connected graphs. Let $G = (V, E)$ be a bridgeless and connected graph. Describe a linear-time algorithm to find a strong orientation of $G$.

{\bfseries Note.} {\em A similar algorithm can be used to find \href{https://en.wikipedia.org/wiki/Ear_decomposition}{ear decompositions}. In the flow networks module, you will describe an algorithm to find $k$-orientations of undirected graphs.}
\end{question}

\begin{solution}

    \textbf{Orientation}
    
    Perform a DFS from a random vertex. 
    For each pair of vertices $u, v$ such that $v$ is ever directly on top of $u$ in the DFS stack,
    let $v$ be a child of $u$ in a imaginary tree that we are NOT constructing in practice. This produces a spanning tree of $G$ called a DFS tree, and this tree $T=(V, F)$.

    A property of $T$ is that for each edge $(a, b)\in E$ such that $(a, b)\notin F$, 
    either $a$ is either an ancestor or a descendant of $b$.
    This is because the DFS traversal always goes as deep as possible, and any cross-edges would have been traversed during a DFS and hence added to $T$.

    For each edge in $T$, orient the edge from parent to child.
    For each edge not in $T$, one is an ancestor of the other, so orient the edge from descendent to ancestor. 

    \textbf{Implementation}

    As for a specific implementation, the edges in $T$ can be oriented during the DFS.
    Whenever a vertex is pushed onto the stack, let the edge between the old top and the new top be oriented from the old top to the new top. 
    During the stage of the DFS when we are checking for unvisited neighbours (that are not the parent), if a neighbour is already visited, then that neighbour is an ancestor (since there are no cross-edges), and thus orient that edge towards the ancestor.
    
    This is just a basic DFS with some constant time operations being performed along the way for each vertex and edge, and therefore this algorithm is linear time.

    \textbf{Correctness}

    There is obviously a path from the root of $T$ to every other vertex, along the oriented graph.
    We want to show that for each vertex in the oriented graph, there exists a path to the root.

    If there is a vertex such that there is no path to the root in the oriented graph, 
    then the vertex from this vertex to its parent would be a bridge vertex.
    This is because cross-edges do not exist, and therefore the only edge in $E$ between the component of the parent, and the component of the child is the edge directly connecting them. 
    This contradicts the fact that the graph is bridgeless.
    Therefore for all vertices, there is a path to the root, and there is a path from the root to every vertex, therefore there is a path from every vertex to every other vertex and hence the oriented graph is a strongly connected graph.
\end{solution}
\end{document}