\documentclass{article}

% Remember to also load the algostyle.sty file into your project.
\usepackage{algostyle}

% Insert new packages here.

\begin{document}
\begin{question}
A {\em dandelion} of length $k$ is an undirected graph that consists of a simple path of length $k$, followed by $k$ additional vertices connected to one of the endpoints.

\begin{figure}[H]
\centering

\begin{tikzpicture}
    \node[circle, draw, right = 0.5cm, red!60!black, very thick] (b) {};
    \node[circle, draw, right = 0.5cm of b, red!60!black, very thick] (c) {};
    \node[circle, draw, right = 0.5cm of c, red!60!black, very thick] (d) {};
    \node[circle, draw, right = 0.5cm of d, red!60!black, very thick] (e) {};
    \node[circle, draw, right = 0.5cm of e, red!60!black, very thick] (f) {};
    
    \node[circle, draw, above left = 0.5cm and 0.25cm of f, green!60!black, very thick] (g) {};
    \node[circle, draw, above right = 0.5cm and 0.25cm of f,green!60!black, very thick] (h) {};
    \node[circle, draw, right = 0.5cm of f, green!60!black, very thick] (i) {};
    \node[circle, draw, below right = 0.5cm and 0.25cm of f, green!60!black, very thick] (j) {};
    \node[circle, draw, below left = 0.5cm and 0.25cm of f, green!60!black, very thick] (k) {};

    % Draw path.
    \draw[red!60!black, very thick] (b) -- (c) -- (d) -- (e) -- (f);
    \draw[green!60!black, very thick] (f) -- (g);
    \draw[green!60!black, very thick] (f) -- (h);
    \draw[green!60!black, very thick] (f) -- (i);
    \draw[green!60!black, very thick] (f) -- (j);
    \draw[green!60!black, very thick] (f) -- (k);
\end{tikzpicture}

\caption*{{\em A dandelion of length $5$.}}
\label{fig:enter-label}
\end{figure}

Given a graph $G$, prove that it is $\mathsf{NP}$-hard to find the length of the longest dandelion subgraph of $G$.

{\bfseries Hint.} {\em $\compproblem{HamiltonianPath}$ is known to be $\mathsf{NP\text-C}$.}
\end{question}

\begin{solution}
Let $\compproblem{Dandelion}$ be the decision problem of determining if there exists 
a dandelion of length $k$ on a graph.
This problem trivially reduces to the corresponding optimisation problem 
of determining the length of the longest dandelion subgraph.

Let $G=(V,E)$ be an instance of $\compproblem{HamiltonianPath}$. Let $n$ be the number of vertices and 
$m$ be the number of edges.
Create a new graph $G'$ which contains all the vertices and edges in $G$,
But each vertex in $V$ is connected to $n + 1$ additional vertices.
Therefore $G'$ has an additional $n(n+1)$ vertices.
Solve $\compproblem{Dandelion}$ on the graph $G'$, where we are trying to find a dandelion 
of length $k = n + 1$. 

If $G$ is a yes instance of $\compproblem{HamiltonianPath}$, 
then there exists a path of length of length $n$ on $G$ which visits each vertex exactly once.
We can extend one end of the path by one, into one of the additional vertices,
to form a path of length $n + 1$. At the other end of the path, that vertex was originally in 
$G$, and therefore has $n+1$ additional vertices attached to it in $G'$.
Therefore there is a dandelion of length $n + 1$ on $G'$, and hence $(G', n+1)$ is 
a yes instance of $\compproblem{Dandelion}$.

Suppose that $(G', n+1)$ is a yes instance of $\compproblem{Dandelion}$.
Note that the longest path that can exist on $G'$ has a length of $n+2$.
It consists of a Hamiltonian Path of $G$, extended by one vertex at both ends.
This path cannot be extended any more since the degree of all of the added vertices are pendant vertices.
Since the diameter of the dandelion of length $n+1$ is $n+2$, 
this means that a path of $n+2$ exists on $G'$, namely the stalk of the dandelion plus one of the seeds.
Therefore $G$ has a Hamiltonian path and hence $G$ is a yes instance of $\compproblem{HamiltonianPath}$.

Therefore $\compproblem{HamiltonianPath}$ reduces to $\compproblem{Dandelion}$ 
which in turn reduces to the optimisation problem in question, and hence the optimisation 
problem is $\mathsf{NP\text{-}H}$.


\end{solution}
\end{document}