\documentclass{article}

% Remember to also load the algostyle.sty file into your project.
\usepackage{algostyle}

% Insert new packages here.

\begin{document}
\begin{question}
Let $G = (V, E)$ be an undirected and unweighted graph. A {\em colouring} of $G$ is an assignment of vertices to colours such that no pair of adjacent vertices share the same colour. This problem is in $\mathsf{NP\text{-}C}$; however, we will come up with an exponential-time dynamic programming. We consider the $k\text{-}\compproblem{Colouring}$ problem, described below.

\begin{itemize}
    \item[] {\bfseries Instance.} An undirected and unweighted graph $G = (V, E)$.
    \item[] {\bfseries Task.} {\em Is there a colouring of $G$ using at most $k$ colours?}
\end{itemize}

\begin{enumerate}[label = (\alph*)]
    \item Describe a brute-force algorithm for $k\text{-}\compproblem{Colouring}$. What is the running time of such an algorithm?

    \item An {\em independent set} of $G$ is a subset of vertices $S \subseteq V$ such that no pair of vertices in $S$ are adjacent. How do independent sets of $G$ relate to colour classes of $G$?

    \item Prove that there exists an optimal $k$-colouring such that a colour class is a maximal independent set.

    We are now ready to describe an ``efficient'' algorithm.

    \item We enumerate over all subsets of $V$. For a subset $S \subseteq V$ of vertices, we define $G[S]$ to be the graph of $G$ {\em induced} by the vertex set $S$.
    
    \begin{enumerate}[label = (\roman*)]
        \item Let $\compproblem{OptColour}(S)$ denote the minimum $k$ such that $G[S]$ is $k$-colourable. Explain why
        \begin{align*}
            \compproblem{OptColour}(S) = 1 + \min\{\compproblem{OptColour}(S \setminus I) : I \text{ is a maximal indep. set in } G[S]\}.
        \end{align*}

        \item What is a suitable base case for this problem?

        \item What is the final solution, and what is the order of computation?

        \item On a graph with $n$ vertices, assume that all maximal independent sets can be generated in time $O^*(3^{n/3})$. Here, $O^*(\cdot)$ omits all polynomial factors; that is, $O^*(a^n) = O(p(n) \cdot a^n)$ where $p(n)$ is a polynomial in $n$.

        Show that $\compproblem{OptColour}(S)$ can be solved in time $O^*\left(3^{\lvert S \rvert /3}\right)$.

        \item Show that the running time is given by $O^*\left((1 + 3^{1/3})^n\right)$. This gives an approximately $O^*(2.4423^n)$ algorithm.

        {\bfseries Hint.} {\em The binomial theorem might come in handy.}
    \end{enumerate}
\end{enumerate}

{\bfseries Note.} {\em This is the algorithm described in \href{https://www.semanticscholar.org/paper/A-Note-on-the-Complexity-of-the-Chromatic-Number-Lawler/0742e3eac4efae7db8c0ac816223e2e4c51a93f6}{[Lawler, '76]}. This was the best known algorithm until a new inclusion-exclusion algorithm was introduced in 2006 by Bj\"orklund and Husfeldt that runs in $O^*(2^n)$; this is the currently best-known algorithm.}
\end{question}

\begin{solution}
\begin{enumerate}[label = (\alph*)]
    \item Solution to part (a) goes here...

    \item Solution to part (b) goes here...

    \item Solution to part (c) goes here...

    \item
    \begin{enumerate}[label = (\roman*)]
        \item Solution to part (i) goes here...

        \item Solution to part (ii) goes here...

        \item Solution to part (iii) goes here...

        \item Solution to part (iv) goes here...

        \item Solution to part (v) goes here...
    \end{enumerate}
\end{enumerate}
\end{solution}
\end{document}