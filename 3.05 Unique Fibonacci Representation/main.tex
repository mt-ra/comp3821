\documentclass{article}

% Remember to also load the algostyle.sty file into your project.
\usepackage{algostyle}

% Insert new packages here.

\begin{document}
\begin{question}
Recall the recursive definition of the Fibonacci sequence: \[F_i = \begin{cases}1 & \text{if } i = 1 \text{ or } i = 2, \\ F_{i - 1} + F_{i - 2} & \text{otherwise.}\end{cases}\]

It turns out that any positive integer can be written as a sum of {\em non-consecutive} Fibonacci numbers; this is known as {\em Zeckendorf's theorem}. For example, we can write the integer 83 as \[83 = 55 + 21 + 5 + 2 = F_{10} + F_8 + F_5 + F_3.\] We will prove this theorem with a greedy algorithm. Let $n$ be a positive integer, and consider the following greedy algorithm.

\begin{itemize}
    \item[] {\em Always choose the largest Fibonacci number that is at most equal to $n$, subtract the integer from $n$, and recurse until there is no remainder.}
\end{itemize}

\begin{enumerate}[label = (\alph*)]
    \item Prove that the algorithm is correct.

    {\bfseries Hint.} {\em You should prove that if you take the largest Fibonacci number $F_k$, then largest Fibonacci number that can fit $n - F_k$ is at most $F_{k - 2}$.}

    \item Prove that the Fibonacci representation produced by the algorithm is {\em unique}.
\end{enumerate}
\end{question}

\begin{solution}
\begin{enumerate}[label = (\alph*)]
    \item Solution to part (a) goes here.

    \item Solution to part (b) goes here.

    \item Solution to part (c) goes here.
\end{enumerate}
\end{solution}
\end{document}