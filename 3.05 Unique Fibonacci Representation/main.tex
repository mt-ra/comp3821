\documentclass{article}

% Remember to also load the algostyle.sty file into your project.
\usepackage{algostyle}

% Insert new packages here.

\begin{document}
\begin{question}
Recall the recursive definition of the Fibonacci sequence: \[F_i = \begin{cases}1 & \text{if } i = 1 \text{ or } i = 2, \\ F_{i - 1} + F_{i - 2} & \text{otherwise.}\end{cases}\]

It turns out that any positive integer can be written as a sum of {\em non-consecutive} Fibonacci numbers; this is known as {\em Zeckendorf's theorem}. For example, we can write the integer 83 as \[83 = 55 + 21 + 5 + 2 = F_{10} + F_8 + F_5 + F_3.\] We will prove this theorem with a greedy algorithm. Let $n$ be a positive integer, and consider the following greedy algorithm.

\begin{itemize}
    \item[] {\em Always choose the largest Fibonacci number that is at most equal to $n$, subtract the integer from $n$, and recurse until there is no remainder.}
\end{itemize}

\begin{enumerate}[label = (\alph*)]
    \item Prove that the algorithm is correct.

    {\bfseries Hint.} {\em You should prove that if you take the largest Fibonacci number $F_k$, then largest Fibonacci number that can fit $n - F_k$ is at most $F_{k - 2}$.}

    \item Prove that the Fibonacci representation produced by the algorithm is {\em unique}.
\end{enumerate}
\end{question}

\begin{solution}
\begin{enumerate}[label = (\alph*)]
    \item 
    Let $F_k$ be the largest Fibonacci number that is less than or equal to $n$.
    We want to show that the largest Fibonacci number less than or equal to $n-F_k$ is at most $F_{k-2}$.

    Assume for the sake of contradiction that the largest Fibonacci number less than or equal to $n-F_k$ is greater than $F_{k-2}$. 
    Therefore the largest Fibonacci number less than or equal to $n-F_k$
    is greater than or equal to $F_{k-1}$. Let this Fibonacci number be $F_j$, where $j\geq k-1$.

    By definition $n - F_k\geq F_j$. Rearranging gives $n \geq F_k + F_j \geq F_k + F_{k-1} \geq F_{k+1}$, which contradicts the fact that $F_k$ is the largest
    Fibonacci number less than or equal to $n$. Therefore the selection of Fibonacci numbers will always be non-consecutive.

    The algorithm terminates when $n = 0$.
    It is guaranteed that $n$ never becomes negative, since the number you subtract from $n$ is always less than or equal to it. 
    There is also no possible scenario where there doesn't exist a valid Fibonacci number less than or equal to $n$, while $n$ is positive, since $F_1$ and $F_2$ are equal to $1$ and thus both satisfy the property that they are less than or equal to any positive $n$. 
    Therefore a selection of non-consecutive Fibonacci numbers will always be produced by this algorithm.
    
    \item First we must prove a theorem regarding the upper bound for the sum of non-consecutive Fibonacci numbers. 
    Define the $\textsc{Sum}$ function such that it returns the sum of elements in a set of integers.
    We want to show that for any set $S$ of non-consecutive Fibonacci numbers, where $F_{n}$ is the largest element in $S$, that 
    $$\textsc{Sum}(S) < F_{n+1}.$$
    This should apply for $n \geq 4$.
    This theorem can be proved with induction.

    Let $S_n$ be the set of non-consecutive Fibonacci numbers, where $F_{n}$ is the largest element, such that the sum of all elements is maximal. 
    For example $S_2=\{1\}$, $S_3=\{2\}$, $S_4=\{1, 3\}$. 

    The base cases are trivial to prove. For the inductive step, assume that 
    $\textsc{Sum}(S_k) < F_{k+1}$ and $\textsc{Sum}(S_{k+1}) < F_{k+2}$ for 
    some positive integer $k$. 
    The sum $\textsc{Sum}(S_{k+2})$ is equal to $F_{k+2}$ plus the 
    maximum possible sum of non-consecutive Fibonacci numbers, 
    with the maximum element being at most $F_{k}$.

    \begin{align*}
    \textsc{Sum}(S_{k+2})&=F_{k+2} + \mathrm{max}(\{\textsc{Sum}(S_i):i\leq k\})\\
    &< F_{k+2} + \mathrm{max}(\{F_{i+1}:i\leq k\})\\
    &< F_{k+2} + F_{k+1}\\
    &< F_{k+3}.
    \end{align*}

    Therefore the statement $\textsc{Sum}(S_n) < F_{n+1}$ is true for all positive integers $n$.

    For any $n$, We want to show that the sequence generated by our algorithm is the only possible sequence of non-consecutive Fibonacci numbers which sum to $n$.
    
    This can be proved by contradiction. 
    Let $A$ be the sequence of non-consecutive Fibonacci numbers summing to $n$, derived using the greedy algorithm, sorted in decreasing order.
    Assume for the sake of contradiction that there exists another decreasing sequence of non-consecutive Fibonacci numbers $B$ which also sums to $n$.

    I will now be using python array slicing notation because I'm silly.
    Since $A\neq B$, there must be a minimal $i$ such that $A[i]\neq B[i]$ (the first point of difference). 
    Up until the first point of difference, the arrays have been identical, $A[1:i]=B[1:i]$. 
    
    Let $s = \textsc{Sum}(A[1:i]) = \textsc{Sum}(B[1:i])$. According to the greedy algorithm, $A[i]$ must equal to the largest Fibonacci number less than or equal to $n-s$. Let this Fibonacci number be $F_k$. On the other hand, $B[i]\neq A[i]$ and so $B[i]$ is at most equal to $F_{k-1}$. 
    Since both sequences sum to $n$,
    \begin{align*}
        n=\textsc{Sum}(A[1:]) &= \textsc{Sum}(B[1:])\\
        \textsc{Sum}(A[1:i]) + A[i] + \textsc{Sum}(A[i+1:]) &= \textsc{Sum}(B[1:i]) + B[i] + \textsc{Sum}(B[i+1:])\\
        F_k + \textsc{Sum}(A[i+1:]) &\leq F_{k-1} + \textsc{Sum}(B[i+1:])\\
        F_k - F_{k-1} &\leq \textsc{Sum}(B[i+1:]) - \textsc{Sum}(A[i+1:])\\
        F_{k-2} &\leq \textsc{Sum}(B[i+1:])
    \end{align*}
    
    Since $B$ was in decreasing order, the largest element in $B[i+1:]$ is at most $F_{k-3}$. 
    It is also a sequence of non-consecutive Fibonacci numbers.
    Therefore from the theorem proved above, $\textsc{Sum}(B[i+1:]) < F_{k-2}$ which is a contradiction.


\end{enumerate}
\end{solution}
\end{document}