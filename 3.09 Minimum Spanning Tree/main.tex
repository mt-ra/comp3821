\documentclass{article}

% Remember to also load the algostyle.sty file into your project.
\usepackage{algostyle}

% Insert new packages here.

\begin{document}
\begin{question}
Let $G = (V, E, w)$ be an undirected and weighted graph, where $w(e) > 0$ is an injective weight function (i.e. all edge weights are unique). You are given that a particular edge $e \in E$ has its weight modified, and are asked to find the new minimum spanning tree. Of course, you {\em could} just recompute the minimum spanning tree but that's boring! We want to do something more clever. In all four cases, assume you are given the actual minimum spanning tree $T$ in advance.

\begin{enumerate}[label = (\alph*)]
    \item Suppose that the edge $e \in T$ has its weight decreased. Explain why $T$ is still the minimum spanning tree.

    {\bfseries Hint.} {\em Is this obvious?}

    \item Suppose that the edge $e \not\in T$ has its weight increased. Explain why $T$ is still the minimum spanning tree.

    {\bfseries Note.} {\em The argument is obvious because all of the edge weights are unique... a more subtle argument needs to be made if the edge weights are not necessarily unique.}

    \item Suppose that the edge $e \in T$ has its weight increased. Describe an $O(m + n)$ algorithm to compute the new minimum spanning tree.

    \item Suppose that the edge $e \not\in T$ has its weight decreased. Describe an $O(m)$ algorithm to compute the new minimum spanning tree.

    {\bfseries Hint.} {\em Either $e$ belongs in the new MST or $e$ doesn't...}
\end{enumerate}
\end{question}

\begin{solution}
\begin{enumerate}[label = (\alph*)]
    \item Suppose the weight of the edge was decreased by $x> 0$. 
    All spanning trees containing $e$ will have their weight decreased by $x$, 
    and all spanning trees not containing $e$ will have their weight unaffected.
    $T$ contains $e$ so it is one of the spanning trees whos weight has decreased by $x$.
    No other spanning tree had their weight decreased more than $T$ had, therefore $T$ is still the minimum spanning tree.

    \item Suppose the weight of the edge was increased by $x > 0$.
    All spanning trees containing $e$ will have their weight increased by $x$,
    and all spanning trees not containing $e$ will have their weight unaffected.
    $T$ does not contain $e$ so its weight is unaffected. 
    No other spanning tree had their weight decreased, therefore $T$ is still the minimum spanning tree.

    \item 
    
    \textbf{Algorithm}
    
    Remove $e$ from $T$ to get $T-\{e\}$. This divides $T$ into two connected components $C_1$ and $C_2$, which are spanning trees of their respective components.
    Perform traversals of the tree from the endpoints of $e$ to determine which component each vertex is a part of.
    Perform a linear scan through $E$ to find the the lightest edge $e'$ connecting $C_1$ and $C_2$.
    Insert this edge into $T-\{e\}$, which should give us the new minimum spanning tree $T'=T\cup\{e'\}-\{e\}$.

    \textbf{Time Complexity}

    Each DFS is $O(n)$, because the number of edges in a tree is equal to the number of vertices minus 1. 
    Then doing a scan of $E$ will be $O(m)$.
    Therefore the time complexity is $O(m + n)$.

    \textbf{Correctness}

    The cycle property states that a spanning tree $S$ is a minimum spanning tree if 
    and only if for all $e\notin S$, $e$ would be the heaviest edge in the cycle created upon adding $e$ to $S$, or in other words, $e$ is the heaviest edge in the cycle in $S\cup\{e\}$.

    We want to show that $T'$ is a minimum spanning tree.
    Let $B$ be the set of edges not in $T$ which bridge the components $C_1$ and $C_2$.
    Similarly, let $B'$ be the set of edges not in $T'$ which bridge the components $C_1$ and $C_2$.
    
    Consider all edges $a\notin T'$ which connect two vertices in the same component, i.e. all edges in $(G - T') - B'$.
    Upon adding each of these edges, the cycle that is formed does not pass through $e$, 
    the edge that had an increase in weight.
    Therefore if these edges obeyed the cycle property prior to the increase in the 
    weight of $e$, then they would also satisfy the cycle property after the change.
    
    Now we have to show that all the edges $a\in B'$ also satisfy the cycle property after the change.

    \textbf{Case 1: } $e\neq e'$
    
    Assume for the sake of contradiction that there exists $a\in B$ such that $a$ is not the heaviest edge in the cycle of $T'\cup\{a\}$. 
    Let the heaviest edge in the cycle in $T'\cup\{a\}$ be $h$, hence $w(h) > w(a) > w(e')$.
    Since the original tree $T$ was a MST, this means that $h$ cannot exist in the cycle in $T\cup\{a\}$, 
    because otherwise $a$ would not obey the cycle property in the original tree. 
    Therefore $h$ must exist in the cycle in $T\cup\{e'\}$.
    However since $e'\neq e$, this means that $e'\notin T$ and thus $w(e') > w(h)$. 
    This is a contradiction. Therefore all edges $a\notin T'$ satsify the cycle property and hence $T'$ is a MST.

    \textbf{Case 2: } $e=e'$

    Assume for the sake of contradiction that there exists $a\in B$ such that $a$ is not the heaviest edge in the cycle of $T'\cup\{a\}$. 
    Therefore $a$ is also not the heaviest edge in the cycle of $T\cup\{a\}$, which is a contradiction, as this means $a$ 
    would not satisfy the cycle property in the original tree. Therefore $T'$ is a MST.
    
    \item 

    \textbf{Algorithm}

    Insert $e$ into $T$. Now there is a cycle in $T\cup \{e\}$, which can be found with a DFS on $T\cup\{e\}$ starting on $e$. 
    If an endpoint of $e$ is seen as a neighbour, then a cycle is found, since we know this cycle must have $e$ in it.
    At this point, the DFS stack will contain all vertices in the cycle.
    Remove the edge $h$ in the cycle with greatest weight by scanning through the DFS stack to find the new tree $T'$.

    \textbf{Time Complexity}

    The DFS on $T\cup\{e\}$ starting at $e$ will take $O(n)$ time as the number of edges is equal to the number of vertices.

    \textbf{Correctness}

    Again we want to show that all edges $a\notin T'$ satisfy the cycle property.
    Consider the edges $a\notin T'$ such that the cycle in $T\cup\{a\}$ does not contain $h$.
    For each $a$, the cycle in $T'\cup\{a\}$ is the same as the cycle in $T\cup\{a\}$,
    therefore this cycle is unaffected by the updated weight and hence $a$ still satisfies the cycle property.
    Now consider the edges $a\notin T'$ such that the cycle in $T\cup\{a\}$ contains $h$.
    The cycle in $T'\cup\{a\}$ is equal to the cycle in $T\cup\{a\}$ union with the cycle in $T\cup\{e\}$ minus $\{h\}$.
    Since $a$ obeyed the cycle property in the old graph, $w(a)>w(h)$, and hence the weight of $a$ 
    will also be greater than the weight of all other edges in the cycle, since $h$ was defined as the 
    edge in the cycle with the largest weight.
    Therefore all edges in $a\notin T'$ satisfy the cycle property, and hence $T'$ is a MST.


\end{enumerate}
\end{solution}
\end{document}