\documentclass{article}

% Remember to also load the algostyle.sty file into your project.
\usepackage{algostyle}

% Insert new packages here.

\begin{document}
\begin{question}
Let $G = (V, E, w)$ be an undirected and weighted graph, where $w(e) > 0$ is an injective weight function (i.e. all edge weights are unique). You are given that a particular edge $e \in E$ has its weight modified, and are asked to find the new minimum spanning tree. Of course, you {\em could} just recompute the minimum spanning tree but that's boring! We want to do something more clever. In all four cases, assume you are given the actual minimum spanning tree $T$ in advance.

\begin{enumerate}[label = (\alph*)]
    \item Suppose that the edge $e \in T$ has its weight decreased. Explain why $T$ is still the minimum spanning tree.

    {\bfseries Hint.} {\em Is this obvious?}

    \item Suppose that the edge $e \not\in T$ has its weight increased. Explain why $T$ is still the minimum spanning tree.

    {\bfseries Note.} {\em The argument is obvious because all of the edge weights are unique... a more subtle argument needs to be made if the edge weights are not necessarily unique.}

    \item Suppose that the edge $e \in T$ has its weight increased. Describe an $O(m + n)$ algorithm to compute the new minimum spanning tree.

    \item Suppose that the edge $e \not\in T$ has its weight decreased. Describe an $O(m)$ algorithm to compute the new minimum spanning tree.

    {\bfseries Hint.} {\em Either $e$ belongs in the new MST or $e$ doesn't...}
\end{enumerate}
\end{question}

\begin{solution}
\begin{enumerate}[label = (\alph*)]
    \item Solution to part (a) goes here.

    \item Solution to part (b) goes here.

    \item Solution to part (c) goes here.
    
    \item Solution to part (d) goes here.
\end{enumerate}
\end{solution}
\end{document}