\documentclass{article}

% Remember to also load the algostyle.sty file into your project.
\usepackage{algostyle}

% Insert new packages here.

\begin{document}
\begin{question}
Let $P_1, \dots, P_k$ be $k$ polynomials, each of degree at least one and suppose that \[\deg(P_1) + \dots + \deg(P_k) = n.\]

\begin{enumerate}[label = (\alph*)]
    \item Describe an $O(kn\log n)$ algorithm to compute the product $P_1(x) \times \dots \times P_k(x)$.

    \item Describe an $O(n \log n \log k)$ algorithm to compute the product $P_1(x) \times \dots \times P_k(x)$.
\end{enumerate}

{\bfseries Hint.} {\em In both problems, use divide and conquer and FFT.}
\end{question}

\begin{rubric}
\begin{itemize}
    \item This task will form part of the portfolio.
    \item Ensure that your argument is clear and keep reworking your solutions until your lab demonstrator is happy with your work.
\end{itemize}
\end{rubric}

\begin{solution}
\begin{enumerate}[label = (\alph*)]
    \item See part (b) as $O(n\log n\log k)\subseteq O(kn\log n)$.


    \item Perform a divide and conquer, at each step halving the number of polynomials we are taking the product of. 
    We first separate the product into two halves, and recursively call the solution on each half.
    This gives us the polynomial product of the left half and the polynomial product of the right half. 
    We then multiply the resultant products together using FFT.
    For the base case where there is only one polynomial in the product, just return that polynomial.

    % fix to justify the time complexity of this algorithm


\end{enumerate}
\end{solution}
\end{document}